\paragraph{Invocation}
The MMT HTTP server can be started by feeding \code{ombase.war} to a servlet container such as Tomcat or Jetty. A particularly easy way to start it is by executing
\begin{center}
\code{java -jar jetty-runner.jar ombase.war}
\end{center}
Jetty-runner is provided in the main directory or available as part of Jetty.

The MMT server will execute the \code{file startup.mmt} from the current directory to initialize itself. In particular, this file should use the command \code{base} and \code{catalog} or \code{ombase}. If logging is switched on, log output will go to standard output.

\paragraph{GET Requests}
Currently only GET requests are implemented.

A GET request to the URI \code{scheme:authority/;?D?Q?R?A} is answered with the result of \code{D?Q?R A} where \code{A} must be an action taking MMT-URI and returning Non-MMT-Object. Spaces in \code{A} must be written as underscores, special characters in \code{A} must be \%-encoded.

If the URI contains less than $3$ $?$, the missing components default to being empty.

Similarly, a GET request to the URI \code{scheme:authority/D?Q?R?A} is answered with the result of \code{D?Q?R A} (where \code{D?Q?R} is resolved relative to the base URI of the server set in the startup script).

\begin{example}
\url{http://localhost/D?Q?R?component_type_present_U}
retrieves the type of \code{D?Q?R} rendered with style \code{U}.
\end{example}