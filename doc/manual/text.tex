\subsection{Module Level}

Intuitively, modules are the named declarations that introduce scopes containing other named declarations (other modules or symbols).

\subsection{Symbol Level}

Intuitively, symbols are the atomic named declarations contained in modules.

\subsection{Object Level}

Intuitively, objects are the unnamed components occurring in named declarations.
This includes in particular the actual {\mmt} objects that occur as the types and definientia.
But for the purposes of the text syntax, it also all other components such as notations and roles.

\subsubsection{Text Notations}

Currently, the best available write-up is in \cite{GIR:mmtlatex:13}.

\subsubsection{Pragmatic Syntax}

Currently, the best available write-up is in \cite{GIR:mmtlatex:13}.

\subsubsection{Parsing}

Currently, the best available write-up is in \cite{GIR:mmtlatex:13}.

\subsection{Structural Delimiters}

Within this document, we display them as $D_{US}$, $D_{RS}$, $D_{GS}$, and $D_{FS}$.

\paragraph{Editor Support}

jEdit supports setting up abbreviations that make entering any character easy.
For example, one can use the following in the \code{mmt} section of jEdit's \code{abbrevs} file:

\begin{lstlisting}
US|$D_{US}$
GS|$D_{GS}$
RS|$D_{RS}$
FS|$D_{FS}$
\end{lstlisting}

In fact, the \code{abbrevs} file provided with \mmt sets up more sophisticated abbreviations that expand to declaration templates and in particular include the delimiters such as

\begin{lstlisting}
�constant\| : $D_{US}$ = $D_{US}$ # $D_{US}$$D_{RS}$
\end{lstlisting}


\renewcommand{\caret}{\lstinline|^|}
ViM supports the insertion of control characters via the following hotkeys:
\begin{compactitem}
 \item group separator (displayed as \caret]) - ctrl+shift+]
 \item record separator (\caret\caret) - ctrl+shift+\caret
 \item unit separator (\caret\_) - ctrl+shift+\_
\end{compactitem}
Note that in command mode one can look up the ASCII code of a character by pressing \code{ga}.

\paragraph{Font Support}

To display \mmt's special delimiters in text interfaces, it is advisable to use a font that has reasonable glyphs for them.
GNU unifont is an example.
