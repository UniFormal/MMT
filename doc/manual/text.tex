\subsection{Module Level}

Intuitively, modules are the named declarations that introduce scopes containing other named declarations (other modules or symbols).

\subsection{Symbol Level}

Intuitively, symbols are the atomic named declarations contained in modules.

\subsection{Object Level}

Intuitively, objects are the unnamed components occurring in named declarations.
This includes in particular the actual {\mmt} objects that occur as the types and definientia.
But for the purposes of the text syntax, it also all other components such as notations and roles.

\subsubsection{Notations}

Currently, the best available write-up is in \cite{GIR:mmtlatex:13}.

\subsubsection{Pragmatic Syntax}

Currently, the best available write-up is in \cite{GIR:mmtlatex:13}.

\subsubsection{Parsing}

Currently, the best available write-up is in \cite{GIR:mmtlatex:13}.

\subsection{Structural Delimiters}

\paragraph{Editor Support}

jEdit

ViM (v. 7.3.429) by default supports insertion of control characters via hotkeys: group separator (displayed as ^]) - ctrl+shift+], record separator (^^) - ctrl+shift+^, unit separator (^_) - ctrl+shift+_. Tip: in command mode one can look up ASCII code of a character by pressing "ga".


\paragraph{Font Support}

GNU unifont
