MMT provides several interfaces for language-specific customization in external implementations.
All extensions have full access to the \mmt controller and may hold arbitrary state and resources of their own.

All extensions must provide a class \code{C} (as a qualified Java class name using dots to separate components), instantiate a certain interface, and have a constructor that takes no arguments.
Extensions are registered by giving the class name \code{C} (see Sect.~\ref{sec:shell:extensions}).
They are instantiated using Java reflection, and it is the user's responsibility to make sure that \code{C} is on the class path at the time of registration.
Extensions are initialized using an \texttt{init} method that takes a \code{List[String]} argument that is provided during registration.

\ednote{describe details of syntax and semantics extensions}

  \begin{itemize}
    \item \api{backend.Plugin} a generic extension that does nothing other than what happens when running its \texttt{init} method; plugins may declare dependencies to other plugins, which will be registered automatically before the present plugin
    \item \api{backend.Foundation}: for registering a foundation in the sense of \cite{RK:mmt:10}. A foundation implements type and equality checking for a certain foundational theory.
    \item \api{backend.Compiler} for compiling external files into \mmt files
    \item \api{backend.QueryTransformer} for translating search queries into \mmt objects
    \item \api{backend.RoleHandler} for adding further role-based processing steps of constants
    \item \api{presentation.Presenter} for rendering \mmt elements and objects
    \item \api{web.ServerPlugin} for adding functions to the \mmt HTTP server
  \end{itemize}

If an extensions maintains its own data structures and auxiliary threads, it must clean up after itself in its \code{destroy} method to avoid memory leaks.

