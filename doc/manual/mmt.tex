\documentclass{article}
\usepackage{graphicx,url,wrapfig,amsmath,amssymb}  %standard packages
\usepackage{amsthm}
\usepackage[hide]{ed/ed} %for ednotes
%\usepackage{xkeyval} %needed for listings package
%\usepackage{listings} %listings
%\usepackage{lstomdoc} %for omdoc listings
%\usepackage{tikz} %diagrams
%\usetikzlibrary{shapes} %needed for tikz package
%\usetikzlibrary{arrows} %needed for tikz package
\usepackage[top=3cm,bottom=3cm,left=4cm,right=3cm]{geometry} %define borders

\usepackage{basics}
\usepackage[nobook]{theorems}
\usepackage{listings}
\usepackage{xmldoc}
\usepackage{xcolor}
\usepackage{paralist}
\usepackage{mmt_new}

\usepackage{local}

\usepackage[bookmarksnumbered,colorlinks,linkcolor=blue,citecolor=red,urlcolor=gray]{hyperref} %pdf hyper links, load this last

\setlength{\hfuzz}{3pt} \hbadness=10001 %make box warning less strict

\title{The {\mmt} Manual}
\author{Florian Rabe \\
Jacobs University Bremen}

\newcommand{\code}[1]{\texttt{#1}}
\newcommand{\omitted}{\op{None}}
\newcommand{\mmturi}{\ensuremath{\op{MMTURI}}}
\newcommand{\mmtqname}{\ensuremath{\op{QName}}}
\newcommand{\mmtname}{\ensuremath{\op{Name}}}
\newcommand{\snippet}[1]{\texttt{#1}}
\newcommand{\defemph}[1]{{\bf #1}}
\newcommand{\translate}[2]{T(#1,#2)}
\newcommand{\render}[3]{R(#1,#2,#3)}
\newcommand{\iPrec}{\op{iPrec}}
\newcommand{\oPrec}{\op{oPrec}}

\begin{document}
\maketitle

\begin{abstract}
This is a permanently out-of-date collection of documentation meant to supplement the published accounts on the \mmt language and system.
It focuses on technical details and is mainly written as a reference, not as an introduction or tutorial.
\end{abstract}

\section{XML Syntax and MMT URIs}\label{sec:syntax}
  The {\mmt} language and its semantics have been described in depth in \cite{RK:mmt:10}. Here we will focus on describing the XML syntax to someone who already has a general understanding of {\mmt}.

The semantics of notations is described in Sect.~\ref{sec:notations}.

\subsection{URIs}

We define three data types for addressing {\mmt} elements: the type {\mmturi} of {\mmt} URIs and the types {\mmtname} and {\mmtqname} of unqualified and qualified {\mmt} names. They are defined by the following grammar:

\begin{center}
\begin{tabular}{|ll@{\tb}c@{\tb}l|}\hline
MMT URI         & \mmturi  & $\bnfas$ & $N\bnfalt M\bnfalt S$    \\
Namespace URI   & $N$      & $\bnfas$ & $\op{URI}$, no query, no fragment  \\
Module URI      & $M$      & $\bnfas$ & $N?\mmtqname \bnfalt ?/\mmtqname$  \\
Symbol URI      & $S$      & $\bnfas$ & $M?\mmtqname \bnfalt ??\mmtqname \bnfalt ??/\mmtqname$ \\
Unqualified name& \mmtname & $\bnfas$ & $pchar^+$ \\
Qualified name  & \mmtqname& $\bnfas$ & $\mmtname (/\mmtname)^*$ \\
  & $\op{URI}$, $pchar$ & & see RFC 3986~\cite{BerFieMas:05} \\
\hline
\end{tabular}
\end{center}

Namespaces $N$ have no semantics and only serve to disambiguate toplevel declarations.
Modules are any kind of named resource that introduces a scope within which other resources (i.e., symbols) are introduced such as signatures, theories, ontologies. Modules may be nested, in which case they have qualified names.
Symbols are any kind of named atomic resource such as constants, functions, predicates, sorts, axioms, theorems. The names of symbols may be qualified as well, which {\mmt} uses to form qualified names for symbols induced by named imports.

We will use $Q$ and $R$ to range over qualified names. In a URI $N[?Q[?R]]$, $N$, $Q$, and $R$ are called the namespace, module name, and symbol name, respectively.
A URI of this form is called \defemph{absolute} if $N$ is an absolute URI. Otherwise, it is called \defemph{relative} or an {\mmturi} \defemph{reference}.
Every absolute MMT URI has exactly one of the following three forms: $N$, $N?Q$, or $N?Q?R$ where $N$ is an absolute URI. Every relative MMT URI has exactly one of the following seven forms: $n$, $n?q$, $n?q?r$, $?/q$, $?/q?r$, $??r$, or $??/r$ where $n$ is a relative (possibly empty) URI.

Note that a $pchar$ may be any character permitted in a URI except for ``/'',  ``?'', ``\#'' ``['', ``]'', and ``\%''. Furtermore, all percent-encoded characters are permitted.

\begin{example}
In this example, we abbreviate \snippet{http://cds.omdoc.org/algebra/algebra.omdoc} with \snippet{A}.

\snippet{O/algebra/algebra.omdoc?monoid?unit} is an absolute symbol URI. It refers to the symbol \snippet{unit} declared in the module \snippet{monoid} declared in the document \snippet{A}.

The $/$-character in the module part separates submodules. \snippet{A?monoid/latex} refers to the module \snippet{latex} declared within the module \snippet{A?monoid} (e.g., a module containing notations to render monoids in Latex syntax).

The $/$-character in the symbol part separates named imports, called structures in MMT. Let \snippet{A?group?mon} refer to an import \snippet{mon} declared within the module \snippet{A?group} that imports the module \snippet{A?monoid}. Then \snippet{A?group?mon/unit} refers to the symbol \snippet{A?monoid?unit} imported via this import. In general, if the symbol part of an MMT URI has $n$ components, then the first $n-1$ must be the names of named imports.
\end{example}

The resolution of an MMT URI reference $u$ against an absolute base URI $U$ is defined as follows:
\begin{enumerate}
	\item $u$ is of the form $n$, $n?q$, or $n?q?r$: $u$ is resolved relative to the namespace $N$ of $U$. (A possible module or symbol name in $U$ are ignored.) If $N'$ is the result of resolving $n$ against $N$ according to RFC 3986, then the resulting MMT URI is $N'$, $N'?q$, or $N'?q?r$, respectively. \\
	Note that in the special case where $n$ is empty, this implies $N'=n$. (Beware that software packages for the URI data type such as in Java 1.5 might implement the obsolete RFC 2396, where empty $d$ was resolved in the same way as $.$.)
	\item $u$ is of the form $?/q$ or $?/q?r$: $u$ is resolved relative to the namespace $N$ and module name $Q$ of $U$. (A possible symbol name in $U$ is ignored. It is an error if $U$ has no module name.) The resolution is $N?Q/q$ or $N?Q/q?r$, respectively.
	\item $u$ is of the form $??r$ or $??/r$ and $U=N?Q?R$. (It is an error if $U$ is a module or document URI.) The resolution is $N?Q?r$ or $N?Q?R/r$, respectively.
\end{enumerate}
\medskip

\begin{example}
Assume a base URI \snippet{http://cds.omdoc.org/algebra/algebra.omdoc?group?mon}. The following table gives examples of resolutions of relative URIs for each of the above six cases. Here we abbreviate \snippet{http://cds.omdoc.org} with \snippet{O}.\smallskip

\begin{tabular}{|l|l|}\hline
URI   & Resolution \\
\hline
\snippet{mathml.omdoc}  &  \snippet{O/algebra/mathml.omdoc} \\
\snippet{?group}  &  \snippet{O/algebra/algebra.omdoc?group} \\
\snippet{../logics/fol/fol.omdoc?fol?and}  &  \snippet{O/logics/fol/fol.omdoc?fol?and} \\
\snippet{?/latex}  &   \snippet{O/algebra/algebra.omdoc?group/latex} \\
\snippet{?/latex?circ}  &   \snippet{O/algebra/algebra.omdoc?group/latex?circ} \\
\snippet{??/unit}  &   \snippet{O/algebra/algebra.omdoc?group?mon/unit} \\
\hline
\end{tabular}
\end{example}
\bigskip

In addition to the above grammars, we introduce the following convention: {\mmturi}s that contain less than two occurrences of $?$, can also be written with $2$ $?s$ by appending $?$. In other words, $n??$ and $N?Q?$ abbreviate $N$ and $N?Q$, respectively.

Thus, (recalling that no $?$-character may occur in URIs that have no query component) every absolute MMT URI can be written uniquely as a $?$-separated triple. The components of this triple are a URI and two $/$-separated lists of strings.

\paragraph{Relationship with URIs}
Every absolute/relative {\mmturi} is also a legal absolute/relative URI. In particular, if we consider an {\mmturi} $N?Q?R$ as a URI, then $Q?R$ is its query component.
Moreover, the {\mmturi} resolution of $n?q?r$ against $D?Q?R$ is identical to the usual resolution of relative URIs.

The situation is more complicated for those relative {\mmturi}s that begin with $?/$ or $??$. Here, the resolution must be implemented separately. This is unavoidable: In order to subsume common practices regarding XML namespaces, the module and symbol name must be put into the query component; but URIs do not permit relative resolution within the query component.

\paragraph{Relationship between OpenMath identifiers and {\mmturi}s.}
Every absolute {\mmturi} is a triple of namespace, module name, and symbol name. This corresponds directly to the cdbase-cd-name triple in OpenMath identifiers \cite{openmath}. Note that {\openmath} use the fragment component when forming URIs from {\openmath} identifiers; {\mmturi}s avoid this because it would preclude efficient retrieval of individual symbols.


\subsection{Document Level Elements}

\begin{elemdescr}
  \elemlabel{omdoc}{a document unit}
  \begin{attdescr}
    \attribute{name}{\mmturi}{the optional name of the unit}
    \attribute{base}{\mmturi}{the base URI for the unit's content, relative to base URI given by parent, empty by default}
  \end{attdescr}
  \children{omdoc* \& xref* \& module*}{}
  \xmlcomment{If this occurs as the root of a document that has a URL, then the name must be omitted or be equal to the last segment of that URL's path.}
\end{elemdescr}

\begin{elemdescr}
  \elemlabel{dref}{a reference to an external document unit}
  \begin{attdescr}
    \attribute{target}{\mmturi}{the referenced document}
  \end{attdescr}
  \children{}{}
\end{elemdescr}

\begin{elemdescr}
  \elemlabel{mref}{a reference to an external document unit}
  \begin{attdescr}
    \attribute{target}{\mmturi}{the referenced module}
  \end{attdescr}
  \children{}{}
\end{elemdescr}

%\begin{elemdescr}
%  \elemlabel{omtext}{text}
%  \begin{attdescr}
%    \attribute{name}{\mmturi}{the optional name of the omtext}
%    \attribute{base}{\mmturi}{the base URI for the omtext's content, relative to the base URI given by the parent, empty by default}
%  \end{attdescr}
%  \children{p* \& TEXT* }{}
%  \xmlcomment{This represents unstructured text such as a paragraph.}
%\end{elemdescr}

\subsection{Module Level Elements}

\begin{elemdescr}
  \elemlabel{theory}{a theory}
  \begin{attdescr}
    \attribute{base}{\mmturi}{the base URI of the module, relative to the base URI given by the parent, empty by default}
    \attribute{name}{\mmtname}{the name of the view}
    \attribute{meta}{\mmturi}{the URI of the optional meta-theory, relative to base URI given by parent}
  \end{attdescr}
  \children{(include* \& symbol*) | definition\{object\}}{}
  \xmlcomment{Here, a symbol can be a constant or a structure.}
  \xmlcomment{Depending on the children, we speak of \emph{declared} and \emph{defined} theories.
  In the latter case, the definiens is a theory expression.}
\end{elemdescr}

\begin{elemdescr}
  \elemlabel{view}{a theory morphism, postulated link}
  \begin{attdescr}
    \attribute{base}{\mmturi}{the base URI of the module, relative to the document base, empty by default}
    \attribute{name}{\mmtname}{the name of the view}
    \attribute{from}{\mmturi}{the domain of the view, relative to base URI given by parent}
    \attribute{to}{\mmturi}{the codomain of the view, relative to base URI given by parent}
  \end{attdescr}
  \children{(include* \& symbol*) | definition\{object\}}{}
  \xmlcomment{The symbols are assignments, i.e., their name must be the same as that of a corresponding symbol in the domain, their types are predetermined, and their definientia required.}
  \xmlcomment{Depending on the children, we speak of \emph{declared} and \emph{defined} views. In the latter case, the definiens is a morphism expression.}
\end{elemdescr}

\begin{elemdescr}
  \elemlabel{style}{a named set of notations}
  \begin{attdescr}[9cm]
    \attribute{base}{\mmturi}{the base URI of the module, relative to the document base, empty by default}
    \attribute{name}{\mmtname}{the name of the style}
    \attribute{for}{\mmturi}{the base URI used for \snippet{for} attributes, relative to the module's base URI, empty by default}
    \attribute{defaults}{\snippet{use} $|$ \snippet{ignore}}{defaults to \snippet{use}, determines the treatment of default notations given in theories}
  \end{attdescr}
  \children{(include* \& notation*) | definition\{notset\}}{}
  \xmlcomment{Definitions are actually not supported yet and only added for symmetry.}
\end{elemdescr}

All module level elements are named. The base URI of a module defaults to the document URI. However, a differing URI may be provided with the \snippet{base} attribute of the module or an ancestor. The latter should only be used in generated documents because it prevents reference by location.

A module with name $n$ and base URI $B$ is addressable via the module URI $B?n$. Module URIs (but not module names) must be unique within a file.

A document unit with name $n$ whose parent is addressable as $D$ is addressable as $D/n$. Document unit names must be unique within a document unit.

\subsection{Symbol Level Elements}

Some symbol level elements are named (includes and notations optionally so). If a symbol with name $n$ occurs in a module with URI $M$, the symbol is addressable via the URI $M?n$.

\begin{elemdescr}
  \elemlabel{include}{inclusion of a theory/view/style into the containing theory/view/styles}
  \begin{attdescr}
    \attribute{from}{\mmturi}{the included module, relative to containing module}
  \end{attdescr}
  \children{}{}
\end{elemdescr}

\begin{elemdescr}
  \elemlabel{constant}{e.g., a sort, function, predicate, judgment, or proof rule}
  \begin{attdescr}
    \attribute{name}{\mmtname}{the name of the constant}
  \end{attdescr}
    \children{alias\{name\} \& type\{object\}? \& definition\{object\}? \& notation\{notation\}? \& role\{string\}}{}
    \xmlcomment{Constants can occur both in theories and views. In the latter case, their type is predetermined and must be omitted, and their definiens is required.}
\end{elemdescr}

\begin{elemdescr}
  \elemlabel{structure}{named instantiation of another theory, definitional link}
  \begin{attdescr}
    \attribute{name}{\mmtname}{the name of the structure}
    \attribute{from}{\mmturi}{the domain of the structure, relative to containing theory}
  \end{attdescr}
  \children{(include* \& symbol*) | definition\{object\}}{}
  \xmlcomment{A can be an assignment to a constant or an assignment to a structure.}
  \xmlcomment{Depending on the children, we speak of \emph{declared} and \emph{defined} structures. In the latter case, the definiens is a morphism expression.} 
\end{elemdescr}

Every notation has a role. The permitted values of \snippet{role} are given in Sect.~\ref{sec:notations}. There are two ways to give notations, direct and via parameters:

\begin{elemdescr}
  \elemlabel{notation}{a notation}
  \begin{attdescr}[8.5cm]
    \attribute{name}{\mmtname}{the optional name of the notation}
    \attribute{for}{\mmturi}{the optional URI to which the notation applies, relative to base URI of style}
    \attribute{role}{string}{the simple role to which the notation applies}
    \attribute{wrap}{\snippet{true | false}}{a flag specifying whether the notation is merged with more specific ones}
    \attribute{precedence}{$\Z^*$}{the optional output precedence}
  \end{attdescr}
  \children{pres}{}
  \xmlcomment{A notation without a name has no URI. \snippet{precedence} may only be given if the role is bracketable.}
\end{elemdescr}

\begin{elemdescr}
  \elemlabel{notation}{a notation}
  \begin{attdescr}[8.5cm]
    \attribute{name}{\mmtname}{the optional name of the notation}
    \attribute{for}{\mmturi}{the optional URI to which the notation applies, relative to base URI of style}
    \attribute{role}{string}{the simple role to which the notation applies}
    \attribute{precedence}{$\Z^*$}{the optional output precedence}
    \attribute{fixity}{string}{the optional fixity: \snippet{pre | post | in | inter | bind}}
    \attribute{application-style}{string}{the optional application style: \snippet{math | lc}}
    \attribute{associativity}{string}{the optional associativity: \snippet{none | left | right}}
    \attribute{implicit}{$\N$}{the number of implicit arguments}
  \end{attdescr}
  \children{}{}
  \xmlcomment{A notation without a name has no URI. \snippet{precedence} may only be given if the role is bracketable.}
\end{elemdescr}

\subsection{Object Level Elements}

The type \snippet{objects} represents {\mmt} terms. These are given as OpenMath objects wrapped in an OMOBJ element. Furthermore, morphism application of $\mu$ to $\omega$ is encoded in OpenMath using the special theory MMT with base $MMT$:
\begin{lstlisting}
<OMA>
  <OMS base="$MMT$" module="mmt" name="morphism-application"/>
  $\mu$
  $\omega$
</OMA>
\end{lstlisting}


\paragraph{Module Expressions}
Objects may denote \mmt theories and morphisms.
Besides OMS elements referring to \mmt theories, theory expressions can arise from, e.g., instantiation, union, and pushout.
Besides OMS elements referring to \mmt views and structures, morphism expressions can arise from, e.g., instantiation, identity, composition, unit, pushout.
We use $MMT$ to abbreviate \lstinline|http://omdoc.org/mmt|.

Link $D?Q$:
\begin{lstlisting}
<OMS base="$D$" module="$Q$"/>
\end{lstlisting}

Composition $\��{\mu_1}{\ldots}{\mu_n}$:
\begin{lstlisting}
<OMA>
  <OMS base="$MMT$" module="mmt" name="composition"/>
  $\mu_1$
  $\ldots$
  $\mu_n$
</OMA>
\end{lstlisting}

Identity $\mmtident{\qT}$:
\begin{lstlisting}
<OMA>
  <OMS base="$MMT$" module="mmt" name="identity"/>
  $T$
</OMA>
\end{lstlisting}

Theories are encoded like links.

\subsection{Resolving MMT URIs}

\paragraph{Document level}
The scheme and authority of a URI $D$ must resolve to some root directory. Then the path of $D$ is resolved relative to that directory. For the resolution of paths, we treat the paths ending in $/$ and those not ending in $/$ as identical.

A path $P$ is resolved relative to the directory $D$ as follows:
\begin{itemize}
  \item $P$ is empty, then resolve to $D$.
	\item If $P=p/P'$ and $p$ is a subdirectory of $D$, then resolve $P'$ relative to $D/p$.
	\item If $P=p/P'$ and $p$ is a file in $D$, then resolve $P'$ relative to the content of that file (which must be an \snippet{omdoc} element).
\end{itemize}

A path $P$ is resolved relative to an \snippet{omdoc} element $O$ as follows:
\begin{itemize}
  \item $P$ is empty, then resolve to $O$.
	\item If $P=p/P'$ and $p$ is the value of a \snippet{name} attribute of an \snippet{omdoc} child of $O$, then resolve $P'$ relative to that child.
\end{itemize}

Note that this makes the file structure transparent in the following sense. Assume a directory $D$ with files $n_1,\ldots,n_r$ containing the respective \snippet{omdoc} element $O_i$. Let $O$ be the file containing
\begin{lstlisting}
<omdoc>
  <omdoc name="$n_1$">
    $O_1$
  </omdoc>
  $\ldots$
  <omdoc name="$n_r$">
    $O_r$
  </omdoc>
</omdoc>
\end{lstlisting}
Then the resolutions of a path relative to $D$ and $O$ are the same.

We can implement this transparency with a standard apache web server: In every directory $D$ add a file \snippet{index.omdoc} containing $O$ as defined above and add a directive in the \snippet{.htaccess} file to serve \snippet{index.omdoc} as the directory listing. (If $D$ should happen to contain a file \snippet{index.omdoc} already, we can pick any other file name for it.) Then apache's path resolution will correspond to the above definition for all paths that do not end in $/$.

The connection to directory listings is stressed if we use this alternative -- semantically equivalent -- definition of $O$:
\begin{lstlisting}
<omdoc>
  <xref target="$n_1$"/>
  $\ldots$
  <xref target="$n_r$"/>
</omdoc>
\end{lstlisting}

Note that it is always legal to split an \snippet{omdoc} file into directories. However, the opposite is only legal if module names are unique across the directory.

\paragraph{Module Level}
A module level URI $D?Q$ is resolved by resolving $D$ to an \snippet{omdoc} element $O$ and then resolving $Q$ relative to it as follows:
\begin{itemize}
	\item If $Q=q$ is a single segment, then resolve to the module with name $q$ in $O$. Here, the modules in $O$ are the module level children of \snippet{omdoc}-descendants of $O$.
	\item If $Q=q/Q'$, then resolve $Q'$ relative to the \snippet{omdoc}-wrapped resolution $q$.
\end{itemize}
Here we assume that there are no \snippet{base} attributes set in $O$.

\paragraph{Symbol Level}
A module level URI $D?Q?R$ is resolved by resolving $D?Q$ to a module level element $M$ and then resolving $R$ relative to it as follows:
\begin{itemize}
	\item If $R=r$ is a single segment, then resolve to the symbol with name $r$ in $M$. Here, the symbols in $M$ are the symbol level children of \snippet{omdoc}-descendants of $M$.
	\item If $R=r/R'$, then resolve $R'$ relative to the domain of the structure with name $r$ in $M$ and translate the result along said structure.
\end{itemize}

Note that the structure of nested \snippet{omdoc} elements is transparent to the resolution of modules and symbols. Thus, the uniqueness of module and symbol names, which is necessary to make resolution well-defined, nested \snippet{omdoc} elements must be disregarded.


\subsection{Notations in Styles}\label{sec:notations}
  \subsubsection{Presentable Expressions}

A \defemph{presentable} {\mmt} expression is any expression produced from any non-terminal symbol of the {\mmt} grammar. Most presentable expressions are characterized by
\begin{itemize}
	\item a role, which refers to the non-terminal symbol from which it is produced,
	\item a components, which is a list of presentable expressions that occur in the first production.
	\item a path, an {\mmturi} identifying or describing the expression (if possible) or its main component,
\end{itemize}
For example, the expression $@(f,a_1,\ldots,a_n)$ for the application of $f$ to arguments $a_1,\ldots,a_n$ is produced using the production $\omega::=@(\omega,\ldots,\omega)$. Its role is \snippet{application}, and its components are $f,a_1,\ldots,a_n$. If $f$ is a constant, then the path is the {\mmturi} of that constant.

\paragraph{Roles and Components}

In the following we list all presentable expressions with their roles and components. We will use $::$ and $nil$ for head-tail composition of lists and $+$ to append an element to the end of a list. For optional parts, a component $[C]$ means that the components is either $C$ or has the special value $\omitted$.

\begin{center}
\begin{tabular}{|l|l|l|}\hline
Expression & Role & Components\\\hline
$\thdeclm{\qT}{[M]}{\theta}$         & \snippet{Theory}          & $\qT::[M]::\theta$ \\
$\vwdeclm{\ql}{\qS}{\qT}{[\mu]}{\sigma}$ & \snippet{View}        & $\ql::\qS::\qT::[\mu]::\sigma$ \\
$\vwdef{\qi}{\qS}{\qT}{\mu}$         & \snippet{DefinedView}     & $\ql::\qS::\qT::\mu::nil$ \\
$\symdd{\qc}{[\tau]}{[\delta]}$      & \snippet{Constant}        & $\qc::[\tau]::[\delta]::nil$ \\
$\impddm{\qi}{\qS}{[\mu]}{\sigma}$   & \snippet{Structure}       & $\qi::\qS::[\mu]::\sigma$ \\
$\dimpdd{\qi}{\qS}{\mu}$             & \snippet{DefinedStructure}& $\qi::\qS::\mu::nil$ \\
$\maps{\qc}{\omega}$                 & \snippet{ConAss}          & $\qc::\omega::nil$ \\
$\maps{\qi}{\mu}$                    & \snippet{StrAss}          & $\qi::\omega::nil$ \\
$\yps:\tau=\delta$                   & \snippet{Variable}        & $\yps::\tau::\delta::nil$ \\
\hline
$\triple{g}{\qT}{\qc}$               & \snippet{constant}        & $g::\qT::\qc::nil$ \\
$\yps$                               & \snippet{variable}        & see below \\
$\triple{g}{\qT}{\qi}$               & \snippet{structure}       & $g::\qT::\qi::nil$ \\
$\mpath{g}{\ql}$                     & \snippet{view}            & $g::\ql::nil$ \\
$\mpath{g}{\qT}$                     & \snippet{theory}          & $g::\qT:nil$ \\
$\hid$                               & \snippet{hidden}          & $nil$ \\
$\oma{\omega_1,\ldots,\omega_n}$     & \snippet{application} (b) & $\omega_1::\ldots::\omega_n::nil$ \\
$\ombind{\omega_1}{\Upsilon}{\omega_2}$ & \snippet{binding} (b)  & $\omega_1::\Upsilon + \omega_2$ \\
$\omattr{\omega_1}{\omega_2}{\omega_3}$ & \snippet{attribution} (b) & $\omega_2::\omega_1::\omega_3::nil$ \\
\hline
toplevel structural expression       & \snippet{Toplevel}        & see below \\
toplevel object expression           & \snippet{toplevel}        & see below \\
\hline
\end{tabular}
\end{center}

\defemph{Bracketable roles} are those for which rendering will produce brackets based on input and output precedences. They are marked with (b). Only notations for those roles may have a \snippet{precedence} attribute, which defaults to $0$ if omitted.

There are two special cases:
\begin{itemize}
	\item A variable occurrence has three components: its name, its de-Bruijn index, and the id of the content expression (\snippet{OMATTR} or \snippet{OMV}) where the variable is bound (see Sect.~\ref{sec:notations:misc} for IDs).
	\item The role \snippet{Toplevel} is chosen for every structural expression immediately after the translation algorithm is called from the outside (as opposed to recursive calls occurring during the translation). Its only component is the expression itself. Using this role, it is possible to include header and footer into the result of the translation.
	\item The role \snippet{toplevel} is similar to the above, but used whenever the toplevel of an object is translated (from the outside or by recursion). This permits to wrap all objects in some way if the output format requires it (e.g., a \snippet{math} element when generating presentation {\mathml}). This role has two components: the object itself and its {\openmath} XML representation. In the latter component, all subexpressions have unique XML IDs (see Sect.~\ref{sec:notations:misc} for IDs).
\end{itemize}

\paragraph{Paths}
The meaning of the path depends on the presentable expression, or more strictly its role. The path is defined as follows:

\begin{itemize}
	\item For all structural roles, the path is the {\mmturi} of the expression, e.g., $\mpath{g}{\qT}$ for a theory $\qT$ declared in a document $g$.
	\item For all roles that refer to an expression, it is the {\mmturi} of that expression. This applies to the roles \snippet{constant}, \snippet{structure}, \snippet{view}, \snippet{theory}.
	\item For roles of composed objects, paths are computed recursively as follows:
    \begin{itemize}
	    \item \snippet{variable}: none
	    \item \snippet{application}: the path of the applied function,
	    \item \snippet{binding}: the path of the binder,
	    \item \snippet{attribution}: the path of the key,
	    \item \snippet{hidden}: none,
	    \item values and foreign objects: none
    \end{itemize}
  \item For the roles \snippet{Toplevel} and \snippet{toplevel}, the path is the path of the toplevel expression.
\end{itemize}

\subsubsection{Syntax and Semantics of Presentations}

The type \snippet{pres} represents presentations. These are lists of \emph{presentation elements} that are used to define structural translations of {\mmt} expressions into other formats or languages. A presentation is evaluated relative to a list of {\mmt} expressions, and this evaluation returns a string or an XML element. Syntax and semantics of presentations are described in Sect.~\ref{sec:notations}.

In the following we will define the well-formed presentation elements and their semantics. We write $[m,n]$ for the set of all integers between and including $m$ and $n$ and $\Z^*$ for the set $\Z\cup\{\snippet{infinity},\snippet{-infinity}\}$.

For a presentation element $P$, its evaluation $\render{P}{C}{i}$ is parametric in a list $C=C_0,\ldots,C_{n-1}$ of {\mmt} expressions and a value $i\in\{0,\ldots,{n-1}\}$. The evaluation returns a string or a list of XML elements.

For a list of presentation elements $P_1\ldots P_n$, the evaluation $\render{P_1\ldots P_n}{C}{i}$ is the concatenation $\render{P_1}{C}{i}+\ldots+\render{P_n}{C}{i}$. If any of these returns an XML element, the concatenation is the concatenation of XML elements where all string components are treated as XML text nodes; consecutive text nodes are merged. Otherwise, it is the concatenation of strings. (This concatenation is associative.)

\paragraph{Producing Literal Values}

\begin{elemdescr}
  \elemlabel{text}{produce a string}
  \begin{attdescr}
    \attribute{value}{string}{the string to produce, defaults to the empty string}
  \end{attdescr}
  \children{}{}
\end{elemdescr}

\begin{evaluation}
This presentation element is evaluated as the value of the \snippet{value} attribute.
\end{evaluation}

\begin{elemdescr}
  \elemlabel{element}{produce an XML element}
  \begin{attdescr}
    \attribute{prefix}{string}{the namespace prefix of the element, default to the empty string}
    \attribute{name}{string}{the label of the element, defaults to the empty string}
  \end{attdescr}
  \children{pres \& attribute*}{}
\end{elemdescr}

\begin{evaluation}
This presentation element is evaluated as the XML element with namespace prefix and label as given by the \snippet{prefix} and \snippet{name} attributes. If the former is empty, the element has no namespace prefix.
The list of children of the produced XML element is the evaluation of the children of the presentation element except for the \snippet{attribute} elements. The attributes of the XML element are given by the evaluation of all the \snippet{attribute} children.
\end{evaluation}

\begin{elemdescr}
  \elemlabel{attribute}{produce an XML attribute}
  \begin{attdescr}
    \attribute{prefix}{string}{the namespace prefix of the attribute, default to the empty string}
    \attribute{name}{string}{the label of the attribute, defaults to the empty string}
  \end{attdescr}
  \children{pres}{}
  \xmlcomment{No child may be an \snippet{element}.}
\end{elemdescr}

\begin{evaluation}
An \snippet{attribute} element is evaluated as the XML attribute with namespace prefix and name as given by the \snippet{prefix} and \snippet{name} attributes. If the former is empty, the attribute has no namespace prefix. The value of the attribute is the evaluation of its children.
\end{evaluation}

%We also define the following short cuts. Note that \snippet{newline} also serves as a system-independent way to produce line endings.
%
%\begin{elemdescr}
%  \elemlabel{newline}{shortcut for a (system-dependent) newline character}
%  \children{}{}
%\end{elemdescr}
%
%\begin{elemdescr}
%  \elemlabel{tab}{shortcut for a tab character}
%  \children{}{}
%\end{elemdescr}


\paragraph{Recursing into Components}

\begin{elemdescr}
  \elemlabel{components}{iterate through $C$}
  \begin{attdescr}
    \attribute{begin}{$\Z$}{the begin index $b$, defaults to $0$}
    \attribute{end}{$\Z$}{the end index $e$, defaults to $-1$}
    \attribute{step}{$\Z\sm\{0\}$}{the step size $s$, defaults to $1$}
  \end{attdescr}
  \children{separator\{pres\}? \& main\{pres\}? }{}
  \xmlcomment{If \snippet{separator} is not present, it defaults to an empty element. If \snippet{main} is not present, it defaults to \snippet{<main><recurse/></main>}.}
\end{elemdescr}

\begin{evaluation}
Let $S$ and $M$ be the list of children of the \snippet{separator} and \snippet{main} elements. Intuitively, this presentation elements evaluates $M$ for all components from $b$ to $e$ with step size $s$, and puts the evaluation of $S$ in between.

Formally, putting $\ov{S}:=\render{S}{C}{i}$, the evaluation is defined as
\[\render{M}{C}{b'}+\ov{S}+\render{M}{C}{b'+s}+\ov{S}+\ldots+\ov{S}+\render{M}{C}{b'+ls}\]
where $b'\in[0,n-1]$, $e'\in[0',n-1]$, and (i) if $s>0$, then $b'\leq e'$ and $l$ is the largest natural number such that $b'+ls\leq e'$, and (ii) if $s<0$, then $b'\geq e'$ and $l$ is the smallest natural number such that $b'+ls\geq e'$.

$b'$ and $e'$ are obtained by the following computation: If $n=0$, or if $b\nin[-n,n-1]$, or $e\nin[-n,n-1]$, an error is issued. Otherwise, $b$ and $e$ are taken modulo $n$ to obtain $b'$ and $e'$. Now if $e-b$ has a different sign from $s$, an error is issued. Then the above conditions hold.
\end{evaluation}

\begin{elemdescr}
  \elemlabel{recurse}{recursively translate a component}
  \begin{attdescr}
    \attribute{precedence}{$\Z^*$}{the relative input precedence for the recursion, defaults to $0$}
    \attribute{offset}{$\Z$}{the offset relative to the current component, defaults to $0$}
  \end{attdescr}
  \children{}{}
  \xmlcomment{A precedence may not be given if within a notation for a structural role.}
\end{elemdescr}

\begin{evaluation}
This presentation element evaluates to $\translate{C_{i+o}}{[p]}$ where $p$ is the value of the \snippet{precedence} attribute and $o$ is the value of the \snippet{offset} attribute.
\end{evaluation}

\begin{example}
For example, let $P$ be the presentation element
\begin{lstlisting}
<components begin="0" end="-1" step="2">
  <separator><newline/></separator>
  <main><text value="Component number "/><index/><text value=": "/><recurse/>
</components>
\end{lstlisting}

Then we have $\render{P}{A::B::C::D::E::F::nil}{i}$ yields
\begin{lstlisting}
Component number 0: $A'$
Component number 2: $C'$
Component number 4: $E'$
\end{lstlisting}
where $A'$, $C'$, and $E'$ denote the recursive renderings of $A$, $C$, and $E$.
\end{example}
\bigskip

\begin{elemdescr}
  \elemlabel{component}{recurse into a single component}
  \begin{attdescr}
    \attribute{index}{$\Z$}{the index of the component}
    \attribute{precedence}{$\Z$}{the relative input precedence of the recursion, defaults to $0$}
  \end{attdescr}
  \children{}{}
  \xmlcomment{A precedence may not be given if within a notation for a structural role.}
\end{elemdescr}

\begin{evaluation}
This is a shortcut for a \snippet{components} elements where \snippet{begin} and \snippet{end} index are $j$ and the only child is \snippet{recurse precedence="$p$"} where $j$ and $p$ are the values of the \snippet{index} and \snippet{precedence} attribute. This means that the evaluation is $\translate{C_j}{[p]}$.
\end{evaluation}

\begin{elemdescr}
  \elemlabel{index}{render the position of a component}
  \begin{attdescr}
    \attribute{offset}{$\Z$}{the offset relative to the current component, defaults to $0$}
  \end{attdescr}
  \children{}{}
\end{elemdescr}

\begin{evaluation}
This presentation element evaluates to the result of $i+o$ as a string where $o$ is the value of the offset attribute.
\end{evaluation}

%\subsubsection{Pattern-matching Components}
%
%\begin{elemdescr}
%  \elemlabel{match}{Select a list of components according to a complex pattern}
%  \begin{attdescr}
%    \attribute{pattern}{\snippet{``.''* \& (``*''\N?)*}}{a sequence of jokers encoding a pattern}
%    \attribute{result}{$[0,l-1]$}{the number of the pattern variable to be selected, where $l$ is the number of jokers}
%    \attribute{step}{\Z}{the step size as for a \snippet{components} element}
%  \end{attdescr}
%  \children{separator\{pres\}? \& main\{pres\}? }{}
%\end{elemdescr}
%
%\begin{evaluation}
%This presentation element is evaluated like a \snippet{components} element except that the processed sublist of components is determined differently.
%
%The value of the \snippet{pattern} attribute is a list of jokers that are matched against $C$:
%\begin{itemize}
%	\item The character ``.'' matches a single component.
%	\item The character ``*'' which matches a list of components.
%\end{itemize}
%The matching of list jokers is such that all list jokers match lists of the same length. If this is not possible remaining components are distributed to the list jokers in the order in which they occur.
%
%Then the value of the \snippet{result} attribute identifies the position of the joker (counting from $0$) that selects the list of components to be rendered.
%
%It is an error if either of the following holds: The length of $C$ is smaller than the number of ``.'' jokers; there are only ``.'' jokers, but less than the length of $C$.
%\end{evaluation}
%
%\begin{example}
%Consider the following presentation item
%\begin{lstlisting}
%<match pattern="..*.*" result="$r$">
%...
%</match>
%\end{lstlisting}
%and a list of components $C=C_0,\ldots,C_{n-1}$. Then $C$ is split into $5$ parts $P_0$, \ldots, $P_4$, one of which is selected by $r\in [0,4]$. We have $P_0=C_0$ and $P_1=C_1$. For the remainder, there are two cases:
%\begin{itemize}
%	\item $n-3$ is even, say $2m$: Then $P_2=C_2,\ldots,C_{m+1}$, $P_3=C_{m+2}$, and $P_4=C_{m+3},\ldots,C_{n-1}$. The length of both $P_2$ and $P_4$ is $m$.
%	\item $n-3$ is odd, say $2m+1$: Then $P_2=C_2,\ldots,C_{m+2}$, $P_3=C_{m+3}$, and $P_4=C_{m+4},\ldots,C_{n-1}$. The length of $P_2$ is $m+1$, and the length of $P_4$ is $m$.
%\end{itemize}
%\end{example}
%
%\begin{example}
%For a more realistic example, consider the operator for matrix composition taking $m+1$ dimensions and then $m$ matrices as examples. Typically, only the latter are displayed, and they can be selected using \snippet{pattern="**" match="1"}.
%\end{example}
%

\paragraph{Miscellaneous Elements}\label{sec:notations:misc}

\begin{elemdescr}
  \elemlabel{id}{produces a unique ID}
  \children{}{}
\end{elemdescr}

\begin{evaluation}
This evaluates to a string that uniquely identifies the currently translated expression. These IDs can be used to create arbitrary unique names required by the target format of the translation.

This ID is equal to the corresponding XML ID in the {\openmath} expression occurring as the second component of the role \snippet{toplevel}. Thus, it can be used for parallel markup links from presentation to content when translation to presentation {\mathml}.
%For binders, this ID is also the third component of all variables bound by the binder. Thus, it can be used to create links from a variable to its binder when translating to, e.g., HTML.
\end{evaluation}

\begin{elemdescr}
  \elemlabel{ifpresent}{case distinction for omitted components}
  \begin{attdescr}
    \attribute{index}{$\Z$}{the index of the tested component}
  \end{attdescr}
  \children{then\{pres\}? else\{pres\}?}{}
  \xmlcomment{If \snippet{then} or \snippet{else} are not present, they default to empty elements.}
\end{elemdescr}

\begin{evaluation}
Let $j$ be the value of the \snippet{index} attribute. If $C_i\neq\_$, this presentation element evaluates to the evaluation of the children of its \snippet{then} child, otherwise to the evaluation of the children of its \snippet{else} child.
\end{evaluation}

\begin{elemdescr}
  \elemlabel{hole}{a placeholder}
  \begin{attdescr}
    \attribute{index}{$\Z$}{the index of the supplied argument within a list of arguments to fill the placeholder}
  \end{attdescr}
  \children{pres}{}
\end{elemdescr}

\begin{evaluation}
This presentation element serves as a placeholder that is replaced during the dynamic computation of presentations. Even it is not filled, it evaluates to the evaluation of its children.
\end{evaluation}

\begin{elemdescr}
  \elemlabel{fragment}{a call to a template}
  \begin{attdescr}
    \attribute{name}{string}{the template name}
  \end{attdescr}
  \children{arg {pres} * | pres}{}
  \xmlcomment{Children $P$ of the form \snippet{pres} abbreviate a single \snippet{arg} child with children $P$.}
\end{elemdescr}

\begin{evaluation}
This presentation element calls another notation, which is used as a template. It is evaluated by obtaining the presentation $p$ for the notation key $(-,\mathtt{fragment:}n)$ (see below) where $n$ is the value of the \snippet{name} attribute. If $p_0,\ldots,p_n$ are the presentations in the children, then this presentation element is evaluated to the evaluation of $p(p_0,\ldots,p_n)$.

The latter notation describes the filling of placeholders. For a presentation $p$ and a list of presentations $p_0,\ldots,p_n$, we write $p(p_1,\ldots,p_n)$ for the presentation arising from $p$ by replacing every placeholder with index $0\leq \leq n$ in $p$ with $p_i$. Placeholders with index $i>n$ or $i<0$ are replaced with their respective children. For example, brackets are given as presentations in which a placeholder indicates the position of the bracketed expression.
\end{evaluation}

\subsubsection{Semantics of Notations}

A notation is a tuple of a notation key and a presentation and possibly and output precedence. The notation key determines when the notation is used, the presentation determines the rendering produced form the notation.

A \defemph{notation key} is a tuple $(u,r)$ of an optional {\mmturi} $u$ and a role $r$. We write $u=-$ if it is omitted. If $r$ is a bracketable role, then the notation must also give an \defemph{output precedence} that is used for bracket generation. The \defemph{presentation} is an expression in a simple language for text or XML output, it may contain references to components of the currently rendered object. For \emph{declarative notations}, the presentation is computed from parameters such as fixity. The presentation has an optional output precedence.

The high-level structure of the presentation algorithm is as follows:
\begin{enumerate}
	\item Input: a presentable expression $E$ with optional {\mmturi} $U$, role $R$, and component list $C$, a style $N$, and an optional input precedence $\iPrec$.
	\item A notation $n$ applicable to $E$ is selected from $N$.
	\item The presentation $p$ of $n$ is obtained.
	\item If $R$ is bracketable, depending $R$, $\iPrec$, and the output precedence of $n$, $P$ is wrapped in brackets yielding $P'$.
	\item Output: the evaluation $\render{P'}{C}{0}$ of $P'$ in context $C$.
\end{enumerate}
All steps are described below.

\paragraph{Selecting Notations}
Styles may have multiple or no notations for the same notation key. The following rules are used:
\begin{itemize}
  \item A style may not locally declare two notations for the same key.
  \item A style has all locally declared notations, and all notations that imported style have, with one exception: Local notations shadow imported declarations for the same key.
  \item If $U$ is omitted, $n$ is the notation with role $R$ in $N$. It is an error if no such notation exists.
	\item If $U$ is given, then
	   \begin{itemize}
	    \item if $N$ has a notation for either $(U,R)$ or for $(-,R)$, then that $n$ is that notation,
	    \item if $N$ has notations for neither $(U,R)$ or $(-,R)$, it is an error.
	    \item if $N$ has notations $n_1$ with presentation $p_1$ for $(U,R)$ and $n_2$ with presentation $p_2$ for $(-,R)$ and $n_2$ has the \snippet{wrap} flag, then $n$ is a notation with presentation $p_2(p_1)$ (see below for applying presentations); the precedence of $n$ is that of $n_1$.
	    \item if $N$ has notations $n_1$ for $(U,R)$ and $n_2$ for $(-,R)$ but $n_2$ does not have the \snippet{wrap} flag, then $n=n_1$.
    \end{itemize}
\end{itemize}

\paragraph{Obtaining Presentations}
If $n$ gives a presentation directly, $P$ is that presentation.

Otherwise, $P$ is computed from the declarative parameters of $n$ as follows:
\begin{enumerate}
  \item The operator is the first component in $C$.
  \item According to the number $i$ of implicit arguments determined by $n$, the following $i$ components are the implicit arguments, the remaining components the explicit arguments.
	\item For pre- and postfix notations, the implicit and the explicit arguments are listed with the separator given by the fragment \snippet{argsep}. Then operator, implicit, and explicit arguments are used to fill the corresponding placeholders in the fragment \snippet{pre} or \snippet{post}, respectively.
	\item For infix notations, the operator and the implicit arguments are used to fill the placeholders in the fragment \snippet{operimp}. Then the result is placed among the explicit arguments using the fragments \snippet{opsep} and \snippet{argsep}.
\end{enumerate}


\paragraph{Bracketing}
If $R$ is not bracketable, no brackets are generated and $P'=P$.

Otherwise, the $P'=b(P,d)$ where $d$ is a below and $b$ is the presentation of one of the following three fragments: \snippet{fragment:brackets}, \snippet{fragment:ebrackets} (elidable brackets), or \snippet{fragment:nobrackets}. While these fragment names a fixed meaning, it is still the style's responsibility to provide notations for them.

The selection among the three fragments is as follows:
\begin{itemize}
	\item If $\iPrec$ is not given, no brackets. Thus, expressions can be forced to be unbrackets by not giving an input precedence. This is typical when recursing from structural levels into object levels.
  \item If $\iPrec$ is given, then the brackets depend on the difference $d=\oPrec-\iPrec$:
  \begin{itemize}
	  \item no brackets if $d=\infty$.
	  \item elidable brackets if $0<d<\infty$.
	  \item brackets if $d\leq 0$.
	\end{itemize}
\end{itemize}

\paragraph{Special Fragment Names}

The following table lists all fragment names that have a special meaning.

\begin{tabular}{|l|p{5cm}|l|}
\hline
Fragment  &  Placeholders & Function \\
\hline
\snippet{brackets} & bracketed expressions & unelidable brackets \\
\snippet{ebrackets} & bracketed expressions, elision level & elidable brackets \\
\snippet{nobrackets} & unbracketed expression & no brackets \\
\snippet{argsep} & none & separator between arguments \\
\snippet{opsep} & none & separator between operator and arguments \\
\snippet{argsep} & none & separator between arguments \\
\snippet{pre}, \snippet{post} & operator, implicit arguments, explicit arguments & pre- and postfix notations \\
\snippet{operimp} & operator, implicit arguments & operator in infix notations \\
\hline
\end{tabular}

\section{Text Syntax}\label{sec:syntax}
  \subsection{Module Level}

Intuitively, modules are the named declarations that introduce scopes containing other named declarations (other modules or symbols).

\subsection{Symbol Level}

Intuitively, symbols are the atomic named declarations contained in modules.

\subsection{Object Level}

Intuitively, objects are the unnamed components occurring in named declarations.
This includes in particular the actual {\mmt} objects that occur as the types and definientia.
But for the purposes of the text syntax, it also all other components such as notations and roles.

\subsubsection{Text Notations}

Currently, the best available write-up is in \cite{GIR:mmtlatex:13}.

\subsubsection{Pragmatic Syntax}

Currently, the best available write-up is in \cite{GIR:mmtlatex:13}.

\subsubsection{Parsing}

Currently, the best available write-up is in \cite{GIR:mmtlatex:13}.

\subsection{Structural Delimiters}

Within this document, we display them as $D_{US}$, $D_{RS}$, $D_{GS}$, and $D_{FS}$.

\paragraph{Editor Support}

jEdit supports setting up abbreviations that make entering any character easy.
For example, one can use the following in the \code{mmt} section of jEdit's \code{abbrevs} file:

\begin{lstlisting}
US|$D_{US}$
GS|$D_{GS}$
RS|$D_{RS}$
FS|$D_{FS}$
\end{lstlisting}

In fact, the \code{abbrevs} file provided with \mmt sets up more sophisticated abbreviations that expand to declaration templates and in particular include the delimiters such as

\begin{lstlisting}
�constant\| : $D_{US}$ = $D_{US}$ # $D_{US}$$D_{RS}$
\end{lstlisting}


\renewcommand{\caret}{\lstinline|^|}
ViM supports the insertion of control characters via the following hotkeys:
\begin{compactitem}
 \item group separator (displayed as \caret]) - ctrl+shift+]
 \item record separator (\caret\caret) - ctrl+shift+\caret
 \item unit separator (\caret\_) - ctrl+shift+\_
\end{compactitem}
Note that in command mode one can look up the ASCII code of a character by pressing \code{ga}.

\paragraph{Font Support}

To display \mmt's special delimiters in text interfaces, it is advisable to use a font that has reasonable glyphs for them.
GNU unifont is an example.


\section{Archives}\label{sec:archives}
  An MMT archive \cite{HIJKR:dimensions:11} organizes connected documents in a file system structure. This is inspired by and similar to the project view in software engineering, specifically in Java. MMT provides archive-level functions for building and indexing document collections.

\paragraph{Dimensions}
MMT supports the following dimensions, which occur as toplevel folders in an archive:
\begin{itemize}
 \item source: source files in any language
 \item compiled: one MMT file for every source file, following the same directory structure
 \item content: one file for every MMT module with their directory structure induced by their MMT URI
 \item narration: one MMT file for every source file, similar to the ones in compiled but containing links into the content instead of MMT modules
 \item relational: relational index
 \item mws: MathWebSearch \cite{mathwebsearch} index
\end{itemize}

\paragraph{Metadata}
Archives may contain meta in a file \texttt{META-INT/MANIFEST.MF}. The syntax of this file is line-wise \texttt{key:value} pairs. Keys may occur only once.
Predefined keys are:
\begin{itemize}
 \item \texttt{id}: a string. This identifies the archive.
 \item \texttt{source}: a string. This identifies the source language and is used to choose a compiler.
 \item \texttt{narration-base}: a URI. URIs for the source/narration documents are formed by concatenating this URI with the path within the archive (excluding the dimension).
 \item \texttt{source-base}: a URI. A URI that is used by compilers as the base URI for source files.
\end{itemize}

\paragraph{Compilers}
Compiler can be provided as plugins to MMT. They are classes that implement the MMT compiler interface (see Sect.~\ref{sec:shell}). This class defines an abstract file-to-file compile method. It also has a method to test applicability: This method is passed the \texttt{source} string above when finding a matching compiler for an archive.

\paragraph{Build Process}
MMT provides to build the dimensions of an archive, starting from the source:
\begin{itemize}
 \item \texttt{compiled} is generated from \texttt{source} using a matching compiler.
 \item \texttt{content} and \texttt{narration} are generated generically from \texttt{content}.
 \item Indices are generated generically from \texttt{content} and \texttt{narration}.
 \item a \texttt{.mar} that packages all dimensions is generated generically.
\end{itemize}



\section{The Catalog}\label{sec:catalog}
  All access to {\mmt} knowledge items is done via URIs, not via URLs.
Therefore, the \emph{catalog} is necessary: It maintains a translation map from URIs to URLs.
It is initially empty and built incrementally.

The most important kind of catalog entry is a \emph{prefix}-entry: A catalog entry $(i,l)$ where $i$ is a URI and $l$ is a URL has the effect that every {\mmt}  URI $l/r$ is translated to $i/r$.

Catalog entries are created automatically or explicitly through archives (see Sect.~\ref{sec:archives}) or explicitly via shell commands (see Sect.~\ref{sec:shell:catalog}).

\section{The Shell Interface}\label{sec:shell}
  The main way to interact with \mmt is through the shell.
After invocation, it can be controlled via STDIN/STDOUT.
It can also be scripted by putting shell commands into files. The conventional file ending is \code{msl} for \mmt scripting language.

The \mmt shell responds with an ``empty'' \mmt instance.
Typically, the first thing to do is to load an \code{msl} that configures \mmt and registers extensions and archives.

The shell is invoked by
\begin{center}
\code{java -cp lib/*;mmt-api.jar info.kwarc.mmt.api.Run}
\end{center}
Further command line parameters are passed to the shell and executed as a command (see below). In particular,
\begin{center}
  \code{java -cp mmt-api.jar info.kwarc.mmt.api.Run file F}
\end{center}
can be used to execute a startup/configuration file \code{F}.
For both commands, Windows and Unix shell-scripts are given in the deploy folder of the \mmt repository.
In particular, users may want to asscociate the \code{msl} file ending with the \code{run-file} scripts.

The \code{-cp} parameter defines the Java classpath. Besides the main file \code{mmt-api.jar}, it must contain the dependencies, in particular the Scala library. All dependencies are available from the folder \code{deploy/lib} \footnote{Note that the classpath is a list of entries separated by \code{;} on Windows and \code{:} on Unix.}

Among the dependencies, not all files are needed all the time:
\begin{itemize}
\item If the {\mmt} web server is used, the classpath must contain \code{tiscaf.jar} \cite{tiscaf}.
\item If an SVN repository is used, the classpath must contain \code{svnkit.jar}.
\item If an {\mmt} extension is used, it (and all its dependencies) must be in the classpath.
\end{itemize}

It is often reasonable to increase the memory available to {\mmt} by using the appropriate Java parameter as in (here: 1024 MB) 
\begin{center}
\code{java -Xmx1024m -cp scala/*;mmt-api.jar info.kwarc.mmt.api.Run}
\end{center}

\subsection{General Syntax}
\emph{Commands} are given by a keyword followed by a whitespace-separated list of arguments and terminated by a newline. Empty lines and lines starting with \code{//} are ignored.

We use the following meta-variables for command arguments:
\begin{itemize}
	\item \code{F}: a file name that is interpreted relative to the current directory,
	\item \code{U}: an MMT-URI that is interpreted relative to the current base URI,
		\item \code{A}: an action that is executed on a path or object.
\end{itemize}
The shell maintains one state variable: a base MMT-URI. All paths are interpreted relative to this base.

\subsection{Basic Commands}\label{sec:shell:basic}
\begin{itemize}
\item \code{log}: Registers a log handler. The most important arguments are \code{console} to log to STDOUT and \code{file F} to log to a file.
\item \code{log+ C}, \code{log- C}: Switch on/off logging of component \code{C}.
\item \code{base U}: Sets the base path to \code{U}.
\item \code{file F}: Reads the file \code{F} and executes every line as a command. If a command causes an error, execution is aborted.
\item \code{exit}: Quits the shell.
\end{itemize}

\subsection{Interacting with Documents}\label{sec:shell:interact}
The shell stores a set of documents that are parsed into abstract data structures and made available for querying.

\begin{itemize}
\item \code{read U}: Retrieves, parses, and stores the document with URI \code{U}.
\item \code{clear}: Deletes all read knowledge items from memory.
\item \code{printAll}, \code{printXML}: Dumps the memory, used for testing.
\item Documents are accessed using actions on MMT-URIs as described below.
\end{itemize}
If the execution of these commands, requires documents that have not been read yet, these are retrieved automatically. This happens, for example, if the read document imports a theory from another document, or if the requested path points to a document that has not been read yet.

\subsection{Archive Commands}\label{sec:shell:archives}
The following commands permit the registration and manipulation of archives:

\begin{itemize}
\item \code{archive add F}: This registers an archive with local root folder \code{F}.
\item \code{archive ID O F}: This executes operation \code{O} on the archive with id \code{ID}, optionally restricted to the folder/file \code{F} (slashes as path separators). Legal values for \code{O} are \code{compile}, \code{content} (which will produce \texttt{content}, \texttt{narration}, \texttt{relational}, and \texttt{notation}), \code{mws}, \code{relational} (which reads the relational index), and \code{notation} (which reads the notation index).
\item \code{archive mar F}: This builds the \texttt{mar} file and stores it in \code{F}.
\end{itemize}

\subsection{Retrieval}\label{sec:shell:actions}
Retrieval commands provide a simple infix syntax to pipe retrieved knowledge items through some typed post-processing operations.

The command $O\;A\;a_1\;\ldots\;a_n$ evaluates $O$ and then applies a further action $A$ with additional arguments $a_i$. This corresponds to $O.A(a_1,\ldots,a_n)$ in OO-programming. These actions may be chained: $O\;A\;a_1\;\ldots\;a_n \;B\;b_1\;\ldots\;b_n$ corresponds to $O.A(a_1,\ldots,a_n).B(b_1,\ldots,b_n)$.
Below, actions are classified according to the type of $O$, which is MMT-URI, MMT-Object, or Non-MMT-Object, and the return type, which is MMT-Object, Non-MMT-object, or Nothing.
\smallskip

\noindent
Actions on MMT-URIs $U$
\begin{itemize}
\item empty action: dereferences $U$ and returns it (MMT-Object)
\item \code{closure}: dereferences $U$ and returns the closure as a self-contained document (MMT-Object)
\item \code{deps xml}: dereferences $U$ and returns its dependency set in XML representation (Non-MMT-Object)
\item \code{deps locutor}: dereferences $U$ and returns its dependency set in locutor representation (Non-MMT-Object)
\end{itemize}

\noindent
Actions on MMT-Objects $O$
\begin{itemize}
\item empty action: returns the text representation of $O$ (Non-MMT-Object)
\item \code{component C}: returns the component of $O$ called \code{C} (MMT-Object)
\item \code{xml}: returns the XML representation of $O$ (Non-MMT-Object)
\item \code{present U}: returns the rendering of $O$ using style \code{U} (Non-MMT-Object)
\end{itemize}
Valid component names \code{C} are in particular \code{type} and \code{definition} if $O$ is a constant.


\noindent
Actions on Non-MMT-Objects, all returning Nothing:
\begin{itemize}
\item empty action: prints to standard output
\item \code{write F}: prints to file \code{F}
\end{itemize}

\begin{example}
The action

\noindent
\code{U/algebra/algebra.omdoc?group closure present O/omdoc/ascii.omdoc?ascii write group.txt}
writes the presentation of the closure of the theory of groups to the file \code{group.txt} using an ASCII-based style.

\noindent
\code{U/algebra/algebra.omdoc?group?inv component type xml}
writes the type type of \code{U/algebra/algebra.omdoc?group?inv} in XML to standard output.
\end{example}

\subsection{Catalog Commands}\label{sec:shell:catalog}
The following catalog commands can be used to add catalog entries explicitly:

\begin{itemize}
%\item \code{mathpath fs F}: This is used to add local working copies to the catalog. It takes a file \code{F} in the locutor (see \url{https://locutor.kwarc.info/})  registry format and creates an entry for every working copy listed in it. The repository URLs are treated as URIs that are translated to the location of the local working copy. (See the example file \code{locutor.xml} file in the distribution.)
%\item \code{ombase F}: This creates a catalog entry for an OMBase server described in \code{F}. See the example file \code{ombase.xml} in the documentation.
\item \code{mathpath local}: This adds an entry for the local file system. URIs of the form \code{file:///U} are translated to themselves.
\item \code{mathpath fs F U}: This adds an entry for the local file system. URIs of the form \code{U/R} are translated to \code{F/R}.
\item \code{mathpath svn U}: This adds an entry for a remote SVN repository. URIs contained in it are translated to themselves.\ednote{revision}
\end{itemize}

\subsection{Server Commands}\label{sec:shell:server}
The MMT HTTP server (see Sect.~\ref{sec:http}) is controlled by the commands
\begin{itemize}
\item \code{server on P}: This starts the server on port \code{P}.
\item \code{server off}: This shuts down the server.
\end{itemize}

\subsection{GUI Commands}
The MMT GUI (see Sect.~\ref{sec:gui}) is controlled by the commands
\begin{itemize}
\item \code{browser on}: This opens a new window with an \mmt browser based Java Swing.
\item \code{browser off}: This closes the window.
\end{itemize}

\subsection{Extension Commands}\label{sec:shell:extensions}
Extensions are registered using commands of the form \texttt{kind C ARGS}. \code{C} gives the qualified class name, and \code{ARGS} is a whitespace-separated list of arguments that is passed to the extension during initialization.

\begin{itemize}
\item \code{extension C ARGS}: This registers the extension with qualified Java class name \code{C}. \code{C} must extend  \texttt{info.kwarc.mmt.api.backend.Extension}. Relevant subinterfaces of \texttt{Extension} are explained in Sect.~\ref{sec:extensions}.
All extensions take a list of string arguments, which is passed as \code{ARGS} where the arguments separated by whitespace.
\end{itemize}

\section{The HTTP Interface}\label{sec:http}
  Besides being an API, the main interface for both humans and machines to the {\mmt} system is via the web server. See Sect.~\ref{sec:shell:server} for how to start and stop the server.

It does not matter whether a request originates from the local or a remote machine.

\subsection{Human Interface}

The {\mmt} web server can be accessed with any browser, e.g., by pointing it to \code{http://localhost:8080} after starting the server with \code{server on 8080} on the shell.

\subsection{Machine Interface}

The server responds to a number of requests that permit software systems to interact with {\mmt}.

\paragraph{GET requests}
A GET request of the path \code{/:mmt?URI\_A} is answered with the result of \code{URI A} where \code{A} is any action taking a MMT-URI. Spaces in \code{A} must be written as underscores, special characters in \code{A} must be \%-encoded.
If the URI contains less than $3$ $?$, the missing components default to being empty.

\begin{example}
\url{http://localhost:8080/:mmt?D?Q?R?component_type_present_U}
retrieves the type of \code{D?Q?R} rendered with style \code{U}.
\end{example}

\paragraph{POST requests}
POST requests are used to access the query \cite{rabe:querying:12} and the computation server.\ednote{add details}

\section{The Graphical Interace}\label{sec:gui}
  The graphical interface provides a view of all archives and their contained documents.

It permits inspecting all content elements using various \api[presentation]{Presenter}s.

\section{The API}
  The API documentation is available in the \texttt{doc} folder of the \mmt repository.

\section{MMT Extensions and Plugins}\label{sec:extensions}
  MMT provides several interfaces for language-specific customization in external implementations.
All extensions have full access to the \mmt controller and may hold arbitrary state and resources of their own.

All extensions must provide a class \code{C} (as a qualified Java class name using dots to separate components), instantiate a certain interface, and have a constructor that takes no arguments.
Extensions are registered by giving the class name \code{C} (see Sect.~\ref{sec:shell:extensions}).
They are instantiated using Java reflection, and it is the user's responsibility to make sure that \code{C} is on the class path at the time of registration.
Extensions are initialized using an \texttt{init} method that takes a \code{List[String]} argument that is provided during registration.

\ednote{describe details of syntax and semantics extensions}

  \begin{itemize}
    \item \api{backend.Plugin} a generic extension that does nothing other than what happens when running its \texttt{init} method; plugins may declare dependencies to other plugins, which will be registered automatically before the present plugin
    \item \api{backend.Foundation}: for registering a foundation in the sense of \cite{RK:mmt:10}. A foundation implements type and equality checking for a certain foundational theory.
    \item \api{backend.Compiler} for compiling external files into \mmt files
    \item \api{backend.QueryTransformer} for translating search queries into \mmt objects
    \item \api{backend.RoleHandler} for adding further role-based processing steps of constants
    \item \api{presentation.Presenter} for rendering \mmt elements and objects
    \item \api{web.ServerPlugin} for adding functions to the \mmt HTTP server
  \end{itemize}

If an extensions maintains its own data structures and auxiliary threads, it must clean up after itself in its \code{destroy} method to avoid memory leaks.




\bibliographystyle{alpha}
\bibliography{bib/systems,bib/pub_rabe,kwarc}
\end{document}