The {\mmt} language and its semantics have been described in depth in \cite{RK:mmt:10}. Here we will focus on describing the XML syntax to someone who already has a general understanding of {\mmt}.

The semantics of notations is described in Sect.~\ref{sec:notations}.

\subsection{URIs}

We define three data types for addressing {\mmt} elements: the type {\mmturi} of {\mmt} URIs and the types {\mmtname} and {\mmtqname} of unqualified and qualified {\mmt} names. They are defined by the following grammar:

\begin{center}
\begin{tabular}{|ll@{\tb}c@{\tb}l|}\hline
MMT URI         & \mmturi  & $\bnfas$ & $N\bnfalt M\bnfalt S$    \\
Namespace URI   & $N$      & $\bnfas$ & $\op{URI}$, no query, no fragment  \\
Module URI      & $M$      & $\bnfas$ & $N?\mmtqname \bnfalt ?/\mmtqname$  \\
Symbol URI      & $S$      & $\bnfas$ & $M?\mmtqname \bnfalt ??\mmtqname \bnfalt ??/\mmtqname$ \\
Unqualified name& \mmtname & $\bnfas$ & $pchar^+$ \\
Qualified name  & \mmtqname& $\bnfas$ & $\mmtname (/\mmtname)^*$ \\
  & $\op{URI}$, $pchar$ & & see RFC 3986~\cite{BerFieMas:05} \\
\hline
\end{tabular}
\end{center}

Namespaces $N$ have no semantics and only serve to disambiguate toplevel declarations.
Modules are any kind of named resource that introduces a scope within which other resources (i.e., symbols) are introduced such as signatures, theories, ontologies. Modules may be nested, in which case they have qualified names.
Symbols are any kind of named atomic resource such as constants, functions, predicates, sorts, axioms, theorems. The names of symbols may be qualified as well, which {\mmt} uses to form qualified names for symbols induced by named imports.

We will use $Q$ and $R$ to range over qualified names. In a URI $N[?Q[?R]]$, $N$, $Q$, and $R$ are called the namespace, module name, and symbol name, respectively.
A URI of this form is called \defemph{absolute} if $N$ is an absolute URI. Otherwise, it is called \defemph{relative} or an {\mmturi} \defemph{reference}.
Every absolute MMT URI has exactly one of the following three forms: $N$, $N?Q$, or $N?Q?R$ where $N$ is an absolute URI. Every relative MMT URI has exactly one of the following seven forms: $n$, $n?q$, $n?q?r$, $?/q$, $?/q?r$, $??r$, or $??/r$ where $n$ is a relative (possibly empty) URI.

Note that a $pchar$ may be any character permitted in a URI except for ``/'',  ``?'', ``\#'' ``['', ``]'', and ``\%''. Furtermore, all percent-encoded characters are permitted.

\begin{example}
In this example, we abbreviate \snippet{http://cds.omdoc.org/algebra/algebra.omdoc} with \snippet{A}.

\snippet{O/algebra/algebra.omdoc?monoid?unit} is an absolute symbol URI. It refers to the symbol \snippet{unit} declared in the module \snippet{monoid} declared in the document \snippet{A}.

The $/$-character in the module part separates submodules. \snippet{A?monoid/latex} refers to the module \snippet{latex} declared within the module \snippet{A?monoid} (e.g., a module containing notations to render monoids in Latex syntax).

The $/$-character in the symbol part separates named imports, called structures in MMT. Let \snippet{A?group?mon} refer to an import \snippet{mon} declared within the module \snippet{A?group} that imports the module \snippet{A?monoid}. Then \snippet{A?group?mon/unit} refers to the symbol \snippet{A?monoid?unit} imported via this import. In general, if the symbol part of an MMT URI has $n$ components, then the first $n-1$ must be the names of named imports.
\end{example}

The resolution of an MMT URI reference $u$ against an absolute base URI $U$ is defined as follows:
\begin{enumerate}
	\item $u$ is of the form $n$, $n?q$, or $n?q?r$: $u$ is resolved relative to the namespace $N$ of $U$. (A possible module or symbol name in $U$ are ignored.) If $N'$ is the result of resolving $n$ against $N$ according to RFC 3986, then the resulting MMT URI is $N'$, $N'?q$, or $N'?q?r$, respectively. \\
	Note that in the special case where $n$ is empty, this implies $N'=n$. (Beware that software packages for the URI data type such as in Java 1.5 might implement the obsolete RFC 2396, where empty $d$ was resolved in the same way as $.$.)
	\item $u$ is of the form $?/q$ or $?/q?r$: $u$ is resolved relative to the namespace $N$ and module name $Q$ of $U$. (A possible symbol name in $U$ is ignored. It is an error if $U$ has no module name.) The resolution is $N?Q/q$ or $N?Q/q?r$, respectively.
	\item $u$ is of the form $??r$ or $??/r$ and $U=N?Q?R$. (It is an error if $U$ is a module or document URI.) The resolution is $N?Q?r$ or $N?Q?R/r$, respectively.
\end{enumerate}
\medskip

\begin{example}
Assume a base URI \snippet{http://cds.omdoc.org/algebra/algebra.omdoc?group?mon}. The following table gives examples of resolutions of relative URIs for each of the above six cases. Here we abbreviate \snippet{http://cds.omdoc.org} with \snippet{O}.\smallskip

\begin{tabular}{|l|l|}\hline
URI   & Resolution \\
\hline
\snippet{mathml.omdoc}  &  \snippet{O/algebra/mathml.omdoc} \\
\snippet{?group}  &  \snippet{O/algebra/algebra.omdoc?group} \\
\snippet{../logics/fol/fol.omdoc?fol?and}  &  \snippet{O/logics/fol/fol.omdoc?fol?and} \\
\snippet{?/latex}  &   \snippet{O/algebra/algebra.omdoc?group/latex} \\
\snippet{?/latex?circ}  &   \snippet{O/algebra/algebra.omdoc?group/latex?circ} \\
\snippet{??/unit}  &   \snippet{O/algebra/algebra.omdoc?group?mon/unit} \\
\hline
\end{tabular}
\end{example}
\bigskip

In addition to the above grammars, we introduce the following convention: {\mmturi}s that contain less than two occurrences of $?$, can also be written with $2$ $?s$ by appending $?$. In other words, $n??$ and $N?Q?$ abbreviate $N$ and $N?Q$, respectively.

Thus, (recalling that no $?$-character may occur in URIs that have no query component) every absolute MMT URI can be written uniquely as a $?$-separated triple. The components of this triple are a URI and two $/$-separated lists of strings.

\paragraph{Relationship between URI and {\mmturi}s.}
Every absolute/relative {\mmturi} is also a legal absolute/relative URI. In particular, if we consider an {\mmturi} $N?Q?R$ as a URI, then $Q?R$ is its query component.
Moreover, the {\mmturi} resolution of $n?q?r$ against $D?Q?R$ is identical to the usual resolution of relative URIs.

The situation is more complicated for those relative {\mmturi}s that begin with $?/$ or $??$. Here, the resolution must be implemented separately. This is unavoidable: In order to subsume common practices regarding XML namespaces, the module and symbol name must be put into the query component; but URIs do not permit relative resolution within the query component.

\paragraph{Relationship between OpenMath identifiers and {\mmturi}s.}
Every absolute {\mmturi} is a triple of namespace, module name, and symbol name. This corresponds directly to the cdbase-cd-name triple in OpenMath identifiers \cite{openmath}. Note that {\openmath} use the fragment component when forming URIs from {\openmath} identifiers; {\mmturi}s avoid this because it would preclude efficient retrieval of individual symbols.


\subsection{Document Level Elements}

\begin{elemdescr}
  \elemlabel{omdoc}{a document unit}
  \begin{attdescr}
    \attribute{name}{\mmturi}{the optional name of the unit}
    \attribute{base}{\mmturi}{the base URI for the unit's content, relative to base URI given by parent, empty by default}
  \end{attdescr}
  \children{omdoc* \& xref* \& module*}{}
  \comment{If this occurs as the root of a document that has a URL, then the name must be omitted or be equal to the last segment of that URL's path.}
\end{elemdescr}

%\begin{elemdescr}
%  \elemlabel{omdoc}{a nondocument unit}
%  \begin{attdescr}
%    \attribute{name}{\mmturi}{the optional name of the unit}
%    \attribute{base}{\mmturi}{the base URI for the document's content, relative to document URI, empty by default}
%  \end{attdescr}
%  \children{omdoc* \& omgroup \& xref* \& module*}{}
%  \comment{If this occurs as the root of a document that has a URL, then the name must be omitted or be equal to the last segment of that URL's path.}
%\end{elemdescr}
%
%\begin{elemdescr}
%  \elemlabel{omtext}{text}
%  \begin{attdescr}
%    \attribute{name}{\mmturi}{the optional name of the omtext}
%    \attribute{base}{\mmturi}{the base URI for the omtext's content, relative to the base URI given by the parent, empty by default}
%  \end{attdescr}
%  \children{p* \& TEXT* }{}
%  \comment{This represents unstructured text such as a paragraph.}
%\end{elemdescr}

\subsection{Module Level Elements}

\begin{elemdescr}
  \elemlabel{theory}{a theory}
  \begin{attdescr}
    \attribute{base}{\mmturi}{the base URI of the module, relative to the base URI given by the parent, empty by default}
    \attribute{name}{\mmtname}{the name of the view}
    \attribute{meta}{\mmturi}{the URI of the optional meta-theory, relative to base URI given by parent}
  \end{attdescr}
  \children{(include* \& symbol*) | definition\{theory\}}{}
  \comment{Here, a symbol can be a constant, a structure, or a notation. Definitions are actually not supported yet and only added for symmetry.}
\end{elemdescr}

\begin{elemdescr}
  \elemlabel{view}{a theory morphism, postulated link}
  \begin{attdescr}
    \attribute{base}{\mmturi}{the base URI of the module, relative to the document base, empty by default}
    \attribute{name}{\mmtname}{the name of the view}
    \attribute{from}{\mmturi}{the domain of the view, relative to base URI given by parent}
    \attribute{to}{\mmturi}{the codomain of the view, relative to base URI given by parent}
  \end{attdescr}
  \children{(include* \& assignment*) | definition\{morph\}}{}
  \comment{An assignment can be an assignment to a constant or an assignment to a structure.}
\end{elemdescr}

\begin{elemdescr}
  \elemlabel{style}{a named set of notations}
  \begin{attdescr}[9cm]
    \attribute{base}{\mmturi}{the base URI of the module, relative to the document base, empty by default}
    \attribute{name}{\mmtname}{the name of the style}
    \attribute{for}{\mmturi}{the base URI used for \snippet{for} attributes, relative to the module's base URI, empty by default}
    \attribute{defaults}{\snippet{use} $|$ \snippet{ignore}}{defaults to \snippet{use}, determines the treatment of default notations given in theories}
  \end{attdescr}
  \children{(include* \& notation*) | definition\{notset\}}{}
  \comment{Definitions are actually not supported yet and only added for symmetry.}
\end{elemdescr}

All module level elements are named. The base URI of a module defaults to the document URI. However, a differing URI may be provided with the \snippet{base} attribute of the module or an ancestor. The latter should only be used in generated documents because it prevents reference by location.

A module with name $n$ and base URI $B$ is addressable via the module URI $B?n$. Module URIs (but not module names) must be unique within a file.

A document unit with name $n$ whose parent is addressable as $D$ is addressable as $D/n$. Document unit names must be unique within a document unit.

\subsection{Symbol Level Elements}

Some symbol level elements are named (notations optionally so). If a symbol with name $n$ occurs in a module with URI $M$, the symbol is addressable via the URI $M?n$.

\subsubsection{Elements in Theories}

\begin{elemdescr}
  \elemlabel{include}{inclusion of a theory into the containing theory}
  \begin{attdescr}
    \attribute{from}{\mmturi}{the domain of the inclusion, relative to containing theory}
  \end{attdescr}
  \children{}{}
\end{elemdescr}

\begin{elemdescr}
  \elemlabel{constant}{e.g., a sort, function, predicate, judgment, or proof rule}
  \begin{attdescr}
    \attribute{name}{\mmtname}{the name of the constant}
  \end{attdescr}
  \children{type\{term\}? \& definition\{term\}?}{}
\end{elemdescr}

\begin{elemdescr}
  \elemlabel{structure}{named instantiation of another theory, definitional link}
  \begin{attdescr}
    \attribute{name}{\mmtname}{the name of the structure}
    \attribute{from}{\mmturi}{the domain of the structure, relative to containing theory}
  \end{attdescr}
  \children{(include* \& assignment*) | definition\{morph\}}{}
  \comment{An assignment can be an assignment to a constant or an assignment to a structure.}
\end{elemdescr}

\begin{elemdescr}
  \elemlabel{alias}{alternative name for another symbol}
  \begin{attdescr}
    \attribute{name}{\mmtname}{the name of the alias}
    \attribute{for}{\mmturi}{the symbol the alias references}
  \end{attdescr}
  \children{}{}
  \comment{Contrary to a defined symbol, an alias is transparent and does not introduce a new symbol.}
\end{elemdescr}

\subsubsection{Elements in Links}

\begin{elemdescr}
  \elemlabel{include}{inclusion of a morphism into the containing link}
  \children{morph}{}  
\end{elemdescr}

\begin{elemdescr}
  \elemlabel{conass}{assignment to a constant in a link}
  \begin{attdescr}
    \attribute{name}{\mmtqname}{the name of the constant}
  \end{attdescr}
  \children{term}{}
  \comment{The child is the term that is assigned to the constant.}
\end{elemdescr}

\begin{elemdescr}
  \elemlabel{strass}{assignment to a structure in a link}
  \begin{attdescr}
    \attribute{name}{\mmtqname}{the name of the structure}
  \end{attdescr}
  \children{morph}{}
  \comment{The child is the morphism that is assigned to the constant.}
\end{elemdescr}

\subsubsection{Elements in Styles}

\begin{elemdescr}
  \elemlabel{include}{inclusion of a style into the containing style}
  \begin{attdescr}
    \attribute{from}{\mmturi}{the domain of the inclusion, relative to base URI of containing style}
  \end{attdescr}
  \children{}{}
\end{elemdescr}

Every notation has a role. The permitted values of \snippet{role} are given in Sect.~\ref{sec:notations}. There are two ways to give notations, direct and via parameters:

\begin{elemdescr}
  \elemlabel{notation}{a notation}
  \begin{attdescr}[8.5cm]
    \attribute{name}{\mmtname}{the optional name of the notation}
    \attribute{for}{\mmturi}{the optional URI to which the notation applies, relative to base URI of style}
    \attribute{role}{string}{the simple role to which the notation applies}
    \attribute{wrap}{\snippet{true | false}}{a flag specifying whether the notation is merged with more specific ones}
    \attribute{precedence}{$\Z^*$}{the optional output precedence}
  \end{attdescr}
  \children{pres}{}
  \comment{A notation without a name has no URI. \snippet{precedence} may only be given if the role is bracketable.}
\end{elemdescr}

\begin{elemdescr}
  \elemlabel{notation}{a notation}
  \begin{attdescr}[8.5cm]
    \attribute{name}{\mmtname}{the optional name of the notation}
    \attribute{for}{\mmturi}{the optional URI to which the notation applies, relative to base URI of style}
    \attribute{role}{string}{the simple role to which the notation applies}
    \attribute{precedence}{$\Z^*$}{the optional output precedence}
    \attribute{fixity}{string}{the optional fixity: \snippet{pre | post | in | inter | bind}}
    \attribute{application-style}{string}{the optional application style: \snippet{math | lc}}
    \attribute{associativity}{string}{the optional associativity: \snippet{none | left | right}}
    \attribute{implicit}{$\N$}{the number of implicit arguments}
  \end{attdescr}
  \children{}{}
  \comment{A notation without a name has no URI. \snippet{precedence} may only be given if the role is bracketable.}
\end{elemdescr}

\subsection{Fragment References}

In addition, document fragments defined elsewhere may be referenced. Documents with references are semantically identical to those where references are replaced with their resolution except that the appropriate \snippet{base} attribute is added after resolving.

\begin{elemdescr}
  \elemlabel{X}{a reference to a X element}
  \begin{attdescr}
    \attribute{href}{\mmturi}{the URI of the referenced element}
  \end{attdescr}
  \children{}{}
  \comment{\snippet{X} may be any document, module, or symbol level label.}
  \comment{These elements may carry additional attributes if their key and value are identical to those in the referenced element.}
\end{elemdescr}

\subsection{Object Level Elements}

\subsubsection{Terms}

The type \snippet{term} represents {\mmt} terms. These are given as OpenMath objects wrapped in an OMOBJ element. Furthermore, morphism application of $\mu$ to $\omega$ is encoded in OpenMath using the special theory MMT with base $MMT$:
\begin{lstlisting}
<OMA>
  <OMS base="$MMT$" module="mmt" name="morphism-application"/>
  $\mu$
  $\omega$
</OMA>
\end{lstlisting}

\subsubsection{Morphisms}

The type \snippet{morph} represents {\mmt} morphisms. Morphisms are links, identity, and composition, which are encoded in OpenMath using the special theory MMT. We use $MMT$ to abbreviate \lstinline|http://omdoc.org/mmt|.

Link $D?Q$:
\begin{lstlisting}
<OMS base="$D$" module="$Q$"/>
\end{lstlisting}

Composition $\��{\mu_1}{\ldots}{\mu_n}$:
\begin{lstlisting}
<OMA>
  <OMS base="$MMT$" module="mmt" name="composition"/>
  $\mu_1$
  $\ldots$
  $\mu_n$
</OMA>
\end{lstlisting}

Identity $\mmtident{\qT}$:
\begin{lstlisting}
<OMA>
  <OMS base="$MMT$" module="mmt" name="identity"/>
  $T$
</OMA>
\end{lstlisting}

Theories are encoded like links.

\subsubsection{Presentations}

The type \snippet{pres} represents presentations. These are lists of \emph{presentation elements} that are used to define structural translations of {\mmt} expressions into other formats or languages. A presentation is evaluated relative to a list of {\mmt} expressions, and this evaluation returns a string or an XML element. Syntax and semantics of presentations are described in Sect.~\ref{sec:notations}.

\subsection{Resolving MMT URIs}

\paragraph{Document level}
The scheme and authority of a URI $D$ must resolve to some root directory. Then the path of $D$ is resolved relative to that directory. For the resolution of paths, we treat the paths ending in $/$ and those not ending in $/$ as identical.

A path $P$ is resolved relative to the directory $D$ as follows:
\begin{itemize}
  \item $P$ is empty, then resolve to $D$.
	\item If $P=p/P'$ and $p$ is a subdirectory of $D$, then resolve $P'$ relative to $D/p$.
	\item If $P=p/P'$ and $p$ is a file in $D$, then resolve $P'$ relative to the content of that file (which must be an \snippet{omdoc} element).
\end{itemize}

A path $P$ is resolved relative to an \snippet{omdoc} element $O$ as follows:
\begin{itemize}
  \item $P$ is empty, then resolve to $O$.
	\item If $P=p/P'$ and $p$ is the value of a \snippet{name} attribute of an \snippet{omdoc} child of $O$, then resolve $P'$ relative to that child.
\end{itemize}

Note that this makes the file structure transparent in the following sense. Assume a directory $D$ with files $n_1,\ldots,n_r$ containing the respective \snippet{omdoc} element $O_i$. Let $O$ be the file containing
\begin{lstlisting}
<omdoc>
  <omdoc name="$n_1$">
    $O_1$
  </omdoc>
  $\ldots$
  <omdoc name="$n_r$">
    $O_r$
  </omdoc>
</omdoc>
\end{lstlisting}
Then the resolutions of a path relative to $D$ and $O$ are the same.

We can implement this transparency with a standard apache web server: In every directory $D$ add a file \snippet{index.omdoc} containing $O$ as defined above and add a directive in the \snippet{.htaccess} file to serve \snippet{index.omdoc} as the directory listing. (If $D$ should happen to contain a file \snippet{index.omdoc} already, we can pick any other file name for it.) Then apache's path resolution will correspond to the above definition for all paths that do not end in $/$.

The connection to directory listings is stressed if we use this alternative -- semantically equivalent -- definition of $O$:
\begin{lstlisting}
<omdoc>
  <xref target="$n_1$"/>
  $\ldots$
  <xref target="$n_r$"/>
</omdoc>
\end{lstlisting}

Note that it is always legal to split an \snippet{omdoc} file into directories. However, the opposite is only legal if module names are unique across the directory.

\paragraph{Module Level}
A module level URI $D?Q$ is resolved by resolving $D$ to an \snippet{omdoc} element $O$ and then resolving $Q$ relative to it as follows:
\begin{itemize}
	\item If $Q=q$ is a single segment, then resolve to the module with name $q$ in $O$. Here, the modules in $O$ are the module level children of \snippet{omdoc}-descendants of $O$.
	\item If $Q=q/Q'$, then resolve $Q'$ relative to the \snippet{omdoc}-wrapped resolution $q$.
\end{itemize}
Here we assume that there are no \snippet{base} attributes set in $O$.

\paragraph{Symbol Level}
A module level URI $D?Q?R$ is resolved by resolving $D?Q$ to a module level element $M$ and then resolving $R$ relative to it as follows:
\begin{itemize}
	\item If $R=r$ is a single segment, then resolve to the symbol with name $r$ in $M$. Here, the symbols in $M$ are the symbol level children of \snippet{omdoc}-descendants of $M$.
	\item If $R=r/R'$, then resolve $R'$ relative to the domain of the structure with name $r$ in $M$ and translate the result along said structure.
\end{itemize}

Note that the structure of nested \snippet{omdoc} elements is transparent to the resolution of modules and symbols. Thus, the uniqueness of module and symbol names, which is necessary to make resolution well-defined, nested \snippet{omdoc} elements must be disregarded.
