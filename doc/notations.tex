\subsection{Presentable Expressions}

A \defemph{presentable} {\mmt} expression is any expression produced from any non-terminal symbol of the {\mmt} grammar. Most presentable expressions are characterized by
\begin{itemize}
	\item a role, which refers to the non-terminal symbol from which it is produced,
	\item a components, which is a list of presentable expressions that occur in the first production.
	\item a path, an {\mmturi} identifying or describing the expression (if possible) or its main component,
\end{itemize}
For example, the expression $@(f,a_1,\ldots,a_n)$ for the application of $f$ to arguments $a_1,\ldots,a_n$ is produced using the production $\omega::=@(\omega,\ldots,\omega)$. Its role is \snippet{application}, and its components are $f,a_1,\ldots,a_n$. If $f$ is a constant, then the path is the {\mmturi} of that constant.

\paragraph{Roles and Components}

In the following we list all presentable expressions with their roles and components. We will use $::$ and $nil$ for head-tail composition of lists and $+$ to append an element to the end of a list. For optional parts, a component $[C]$ means that the components is either $C$ or has the special value $\omitted$.

\begin{center}
\begin{tabular}{|l|l|l|}\hline
Expression & Role & Components\\\hline
$\thdeclm{\qT}{[M]}{\theta}$         & \snippet{Theory}          & $\qT::[M]::\theta$ \\
$\vwdeclm{\ql}{\qS}{\qT}{[\mu]}{\sigma}$ & \snippet{View}        & $\ql::\qS::\qT::[\mu]::\sigma$ \\
$\vwdef{\qi}{\qS}{\qT}{\mu}$         & \snippet{DefinedView}     & $\ql::\qS::\qT::\mu::nil$ \\
$\symdd{\qc}{[\tau]}{[\delta]}$      & \snippet{Constant}        & $\qc::[\tau]::[\delta]::nil$ \\
$\impddm{\qi}{\qS}{[\mu]}{\sigma}$   & \snippet{Structure}       & $\qi::\qS::[\mu]::\sigma$ \\
$\dimpdd{\qi}{\qS}{\mu}$             & \snippet{DefinedStructure}& $\qi::\qS::\mu::nil$ \\
$\maps{\qc}{\omega}$                 & \snippet{ConAss}          & $\qc::\omega::nil$ \\
$\maps{\qi}{\mu}$                    & \snippet{StrAss}          & $\qi::\omega::nil$ \\
$\yps:\tau=\delta$                   & \snippet{Variable}        & $\yps::\tau::\delta::nil$ \\
\hline
$\triple{g}{\qT}{\qc}$               & \snippet{constant}        & $g::\qT::\qc::nil$ \\
$\yps$                               & \snippet{variable}        & see below \\
$\triple{g}{\qT}{\qi}$               & \snippet{structure}       & $g::\qT::\qi::nil$ \\
$\mpath{g}{\ql}$                     & \snippet{view}            & $g::\ql::nil$ \\
$\mpath{g}{\qT}$                     & \snippet{theory}          & $g::\qT:nil$ \\
$\hid$                               & \snippet{hidden}          & $nil$ \\
$\ma{\omega}{\mu}$                   & \snippet{morphism-application} (b) & $\omega::\mu::nil$ \\
$\oma{\omega_1,\ldots,\omega_n}$     & \snippet{application} (b) & $\omega_1::\ldots::\omega_n::nil$ \\
$\ombind{\omega_1}{\Upsilon}{\omega_2}$ & \snippet{binding} (b)  & $\omega_1::\Upsilon + \omega_2$ \\
$\omattr{\omega_1}{\omega_2}{\omega_3}$ & \snippet{attribution} (b) & $\omega_2::\omega_1::\omega_3::nil$ \\
$\mu_1\bullet\ldots\bullet \mu_n$    & \snippet{composition} (b) & $\mu_1::\ldots::\mu_n::nil$ \\
$\mmtident{\qT}$                       & \snippet{identity} (b)  & $\qT::nil$ \\
\hline
toplevel structural expression       & \snippet{Toplevel}        & see below \\
toplevel object expression           & \snippet{toplevel}        & see below \\
\hline
\end{tabular}
\end{center}

\defemph{Bracketable roles} are those for which rendering will produce brackets based on input and output precedences. They are marked with (b). Only notations for those roles may have a \snippet{precedence} attribute, which defaults to $0$ if omitted.

There are two special cases:
\begin{itemize}
	\item A variable occurrence has three components: its name, its de-Bruijn index, and the id of the content expression (\snippet{OMATTR} or \snippet{OMV}) where the variable is bound (see Sect.~\ref{sec:notations:misc} for IDs).
	\item The role \snippet{Toplevel} is chosen for every structural expression immediately after the translation algorithm is called from the outside (as opposed to recursive calls occurring during the translation). Its only component is the expression itself. Using this role, it is possible to include header and footer into the result of the translation.
	\item The role \snippet{toplevel} is similar to the above, but used whenever the toplevel of an object is translated (from the outside or by recursion). This permits to wrap all objects in some way if the output format requires it (e.g., a \snippet{math} element when generating presentation {\mathml}). This role has two components: the object itself and its {\openmath} XML representation. In the latter component, all subexpressions have unique XML IDs (see Sect.~\ref{sec:notations:misc} for IDs).
\end{itemize}

\paragraph{Paths}
The meaning of the path depends on the presentable expression, or more strictly its role. The path is defined as follows:

\begin{itemize}
	\item For all structural roles, the path is the {\mmturi} of the expression, e.g., $\mpath{g}{\qT}$ for a theory $\qT$ declared in a document $g$.
	\item For all roles that refer to an expression, it is the {\mmturi} of that expression. This applies to the roles \snippet{constant}, \snippet{structure}, \snippet{view}, \snippet{theory}.
	\item For roles of composed objects, paths are computed recursively as follows:
    \begin{itemize}
	    \item \snippet{variable}: none
	    \item \snippet{morphism-application}: the path of the applied morphism,
	    \item \snippet{application}: the path of the applied function,
	    \item \snippet{binding}: the path of the binder,
	    \item \snippet{attribution}: the path of the key,
	    \item \snippet{identity}: the path of the theory,
	    \item \snippet{hidden}: none,
	    \item \snippet{composition}: The path of the first morphism,
	    \item values and foreign objects: none
    \end{itemize}
  \item For the roles \snippet{Toplevel} and \snippet{toplevel}, the path is the path of the toplevel expression.
\end{itemize}

\subsection{Syntax and Semantics of Presentations}

In the following we will define the well-formed presentation elements and their semantics. We write $[m,n]$ for the set of all integers between and including $m$ and $n$ and $\Z^*$ for the set $\Z\cup\{\snippet{infinity},\snippet{-infinity}\}$.

For a presentation element $P$, its evaluation $\render{P}{C}{i}$ is parametric in a list $C=C_0,\ldots,C_{n-1}$ of {\mmt} expressions and a value $i\in\{0,\ldots,{n-1}\}$. The evaluation returns a string or a list of XML elements.

For a list of presentation elements $P_1\ldots P_n$, the evaluation $\render{P_1\ldots P_n}{C}{i}$ is the concatenation $\render{P_1}{C}{i}+\ldots+\render{P_n}{C}{i}$. If any of these returns an XML element, the concatenation is the concatenation of XML elements where all string components are treated as XML text nodes; consecutive text nodes are merged. Otherwise, it is the concatenation of strings. (This concatenation is associative.)

\subsubsection{Producing Literal Values}

\begin{elemdescr}
  \elemlabel{text}{produce a string}
  \begin{attdescr}
    \attribute{value}{string}{the string to produce, defaults to the empty string}
  \end{attdescr}
  \children{}{}
\end{elemdescr}

\begin{evaluation}
This presentation element is evaluated as the value of the \snippet{value} attribute.
\end{evaluation}

\begin{elemdescr}
  \elemlabel{element}{produce an XML element}
  \begin{attdescr}
    \attribute{prefix}{string}{the namespace prefix of the element, default to the empty string}
    \attribute{name}{string}{the label of the element, defaults to the empty string}
  \end{attdescr}
  \children{pres \& attribute*}{}
\end{elemdescr}

\begin{evaluation}
This presentation element is evaluated as the XML element with namespace prefix and label as given by the \snippet{prefix} and \snippet{name} attributes. If the former is empty, the element has no namespace prefix.
The list of children of the produced XML element is the evaluation of the children of the presentation element except for the \snippet{attribute} elements. The attributes of the XML element are given by the evaluation of all the \snippet{attribute} children.
\end{evaluation}

\begin{elemdescr}
  \elemlabel{attribute}{produce an XML attribute}
  \begin{attdescr}
    \attribute{prefix}{string}{the namespace prefix of the attribute, default to the empty string}
    \attribute{name}{string}{the label of the attribute, defaults to the empty string}
  \end{attdescr}
  \children{pres}{}
  \comment{No child may be an \snippet{element}.}
\end{elemdescr}

\begin{evaluation}
An \snippet{attribute} element is evaluated as the XML attribute with namespace prefix and name as given by the \snippet{prefix} and \snippet{name} attributes. If the former is empty, the attribute has no namespace prefix. The value of the attribute is the evaluation of its children.
\end{evaluation}

%We also define the following short cuts. Note that \snippet{newline} also serves as a system-independent way to produce line endings.
%
%\begin{elemdescr}
%  \elemlabel{newline}{shortcut for a (system-dependent) newline character}
%  \children{}{}
%\end{elemdescr}
%
%\begin{elemdescr}
%  \elemlabel{tab}{shortcut for a tab character}
%  \children{}{}
%\end{elemdescr}


\subsubsection{Recursing into Components}

\begin{elemdescr}
  \elemlabel{components}{iterate through $C$}
  \begin{attdescr}
    \attribute{begin}{$\Z$}{the begin index $b$, defaults to $0$}
    \attribute{end}{$\Z$}{the end index $e$, defaults to $-1$}
    \attribute{step}{$\Z\sm\{0\}$}{the step size $s$, defaults to $1$}
  \end{attdescr}
  \children{separator\{pres\}? \& main\{pres\}? }{}
  \comment{If \snippet{separator} is not present, it defaults to an empty element. If \snippet{main} is not present, it defaults to \snippet{<main><recurse/></main>}.}
\end{elemdescr}

\begin{evaluation}
Let $S$ and $M$ be the list of children of the \snippet{separator} and \snippet{main} elements. Intuitively, this presentation elements evaluates $M$ for all components from $b$ to $e$ with step size $s$, and puts the evaluation of $S$ in between.

Formally, putting $\ov{S}:=\render{S}{C}{i}$, the evaluation is defined as
\[\render{M}{C}{b'}+\ov{S}+\render{M}{C}{b'+s}+\ov{S}+\ldots+\ov{S}+\render{M}{C}{b'+ls}\]
where $b'\in[0,n-1]$, $e'\in[0',n-1]$, and (i) if $s>0$, then $b'\leq e'$ and $l$ is the largest natural number such that $b'+ls\leq e'$, and (ii) if $s<0$, then $b'\geq e'$ and $l$ is the smallest natural number such that $b'+ls\geq e'$.

$b'$ and $e'$ are obtained by the following computation: If $n=0$, or if $b\nin[-n,n-1]$, or $e\nin[-n,n-1]$, an error is issued. Otherwise, $b$ and $e$ are taken modulo $n$ to obtain $b'$ and $e'$. Now if $e-b$ has a different sign from $s$, an error is issued. Then the above conditions hold.
\end{evaluation}

\begin{elemdescr}
  \elemlabel{recurse}{recursively translate a component}
  \begin{attdescr}
    \attribute{precedence}{$\Z^*$}{the relative input precedence for the recursion, defaults to $0$}
    \attribute{offset}{$\Z$}{the offset relative to the current component, defaults to $0$}
  \end{attdescr}
  \children{}{}
  \comment{A precedence may not be given if within a notation for a structural role.}
\end{elemdescr}

\begin{evaluation}
This presentation element evaluates to $\translate{C_{i+o}}{[p]}$ where $p$ is the value of the \snippet{precedence} attribute and $o$ is the value of the \snippet{offset} attribute.
\end{evaluation}

\begin{example}
For example, let $P$ be the presentation element
\begin{lstlisting}
<components begin="0" end="-1" step="2">
  <separator><newline/></separator>
  <main><text value="Component number "/><index/><text value=": "/><recurse/>
</components>
\end{lstlisting}

Then we have $\render{P}{A::B::C::D::E::F::nil}{i}$ yields
\begin{lstlisting}
Component number 0: $A'$
Component number 2: $C'$
Component number 4: $E'$
\end{lstlisting}
where $A'$, $C'$, and $E'$ denote the recursive renderings of $A$, $C$, and $E$.
\end{example}
\bigskip

\begin{elemdescr}
  \elemlabel{component}{recurse into a single component}
  \begin{attdescr}
    \attribute{index}{$\Z$}{the index of the component}
    \attribute{precedence}{$\Z$}{the relative input precedence of the recursion, defaults to $0$}
  \end{attdescr}
  \children{}{}
  \comment{A precedence may not be given if within a notation for a structural role.}
\end{elemdescr}

\begin{evaluation}
This is a shortcut for a \snippet{components} elements where \snippet{begin} and \snippet{end} index are $j$ and the only child is \snippet{recurse precedence="$p$"} where $j$ and $p$ are the values of the \snippet{index} and \snippet{precedence} attribute. This means that the evaluation is $\translate{C_j}{[p]}$.
\end{evaluation}

\begin{elemdescr}
  \elemlabel{index}{render the position of a component}
  \begin{attdescr}
    \attribute{offset}{$\Z$}{the offset relative to the current component, defaults to $0$}
  \end{attdescr}
  \children{}{}
\end{elemdescr}

\begin{evaluation}
This presentation element evaluates to the result of $i+o$ as a string where $o$ is the value of the offset attribute.
\end{evaluation}

%\subsubsection{Pattern-matching Components}
%
%\begin{elemdescr}
%  \elemlabel{match}{Select a list of components according to a complex pattern}
%  \begin{attdescr}
%    \attribute{pattern}{\snippet{``.''* \& (``*''\N?)*}}{a sequence of jokers encoding a pattern}
%    \attribute{result}{$[0,l-1]$}{the number of the pattern variable to be selected, where $l$ is the number of jokers}
%    \attribute{step}{\Z}{the step size as for a \snippet{components} element}
%  \end{attdescr}
%  \children{separator\{pres\}? \& main\{pres\}? }{}
%\end{elemdescr}
%
%\begin{evaluation}
%This presentation element is evaluated like a \snippet{components} element except that the processed sublist of components is determined differently.
%
%The value of the \snippet{pattern} attribute is a list of jokers that are matched against $C$:
%\begin{itemize}
%	\item The character ``.'' matches a single component.
%	\item The character ``*'' which matches a list of components.
%\end{itemize}
%The matching of list jokers is such that all list jokers match lists of the same length. If this is not possible remaining components are distributed to the list jokers in the order in which they occur.
%
%Then the value of the \snippet{result} attribute identifies the position of the joker (counting from $0$) that selects the list of components to be rendered.
%
%It is an error if either of the following holds: The length of $C$ is smaller than the number of ``.'' jokers; there are only ``.'' jokers, but less than the length of $C$.
%\end{evaluation}
%
%\begin{example}
%Consider the following presentation item
%\begin{lstlisting}
%<match pattern="..*.*" result="$r$">
%...
%</match>
%\end{lstlisting}
%and a list of components $C=C_0,\ldots,C_{n-1}$. Then $C$ is split into $5$ parts $P_0$, \ldots, $P_4$, one of which is selected by $r\in [0,4]$. We have $P_0=C_0$ and $P_1=C_1$. For the remainder, there are two cases:
%\begin{itemize}
%	\item $n-3$ is even, say $2m$: Then $P_2=C_2,\ldots,C_{m+1}$, $P_3=C_{m+2}$, and $P_4=C_{m+3},\ldots,C_{n-1}$. The length of both $P_2$ and $P_4$ is $m$.
%	\item $n-3$ is odd, say $2m+1$: Then $P_2=C_2,\ldots,C_{m+2}$, $P_3=C_{m+3}$, and $P_4=C_{m+4},\ldots,C_{n-1}$. The length of $P_2$ is $m+1$, and the length of $P_4$ is $m$.
%\end{itemize}
%\end{example}
%
%\begin{example}
%For a more realistic example, consider the operator for matrix composition taking $m+1$ dimensions and then $m$ matrices as examples. Typically, only the latter are displayed, and they can be selected using \snippet{pattern="**" match="1"}.
%\end{example}
%

\subsubsection{Miscellaneous Elements}\label{sec:notations:misc}

\begin{elemdescr}
  \elemlabel{id}{produces a unique ID}
  \children{}{}
\end{elemdescr}

\begin{evaluation}
This evaluates to a string that uniquely identifies the currently translated expression. These IDs can be used to create arbitrary unique names required by the target format of the translation.

This ID is equal to the corresponding XML ID in the {\openmath} expression occurring as the second component of the role \snippet{toplevel}. Thus, it can be used for parallel markup links from presentation to content when translation to presentation {\mathml}.
%For binders, this ID is also the third component of all variables bound by the binder. Thus, it can be used to create links from a variable to its binder when translating to, e.g., HTML.
\end{evaluation}

\begin{elemdescr}
  \elemlabel{ifpresent}{case distinction for omitted components}
  \begin{attdescr}
    \attribute{index}{$\Z$}{the index of the tested component}
  \end{attdescr}
  \children{then\{pres\}? else\{pres\}?}{}
  \comment{If \snippet{then} or \snippet{else} are not present, they default to empty elements.}
\end{elemdescr}

\begin{evaluation}
Let $j$ be the value of the \snippet{index} attribute. If $C_i\neq\_$, this presentation element evaluates to the evaluation of the children of its \snippet{then} child, otherwise to the evaluation of the children of its \snippet{else} child.
\end{evaluation}

\begin{elemdescr}
  \elemlabel{hole}{a placeholder}
  \begin{attdescr}
    \attribute{index}{$\Z$}{the index of the supplied argument within a list of arguments to fill the placeholder}
  \end{attdescr}
  \children{pres}{}
\end{elemdescr}

\begin{evaluation}
This presentation element serves as a placeholder that is replaced during the dynamic computation of presentations. Even it is not filled, it evaluates to the evaluation of its children.
\end{evaluation}

\begin{elemdescr}
  \elemlabel{fragment}{a call to a template}
  \begin{attdescr}
    \attribute{name}{string}{the template name}
  \end{attdescr}
  \children{arg {pres} * | pres}{}
  \comment{Children $P$ of the form \snippet{pres} abbreviate a single \snippet{arg} child with children $P$.}
\end{elemdescr}

\begin{evaluation}
This presentation element calls another notation, which is used as a template. It is evaluated by obtaining the presentation $p$ for the notation key $(-,\mathtt{fragment:}n)$ (see below) where $n$ is the value of the \snippet{name} attribute. If $p_0,\ldots,p_n$ are the presentations in the children, then this presentation element is evaluated to the evaluation of $p(p_0,\ldots,p_n)$.

The latter notation describes the filling of placeholders. For a presentation $p$ and a list of presentations $p_0,\ldots,p_n$, we write $p(p_1,\ldots,p_n)$ for the presentation arising from $p$ by replacing every placeholder with index $0\leq \leq n$ in $p$ with $p_i$. Placeholders with index $i>n$ or $i<0$ are replaced with their respective children. For example, brackets are given as presentations in which a placeholder indicates the position of the bracketed expression.
\end{evaluation}

\subsection{Semantics of Notations}

A notation is a tuple of a notation key and a presentation and possibly and output precedence. The notation key determines when the notation is used, the presentation determines the rendering produced form the notation.

A \defemph{notation key} is a tuple $(u,r)$ of an optional {\mmturi} $u$ and a role $r$. We write $u=-$ if it is omitted. If $r$ is a bracketable role, then the notation must also give an \defemph{output precedence} that is used for bracket generation. The \defemph{presentation} is an expression in a simple language for text or XML output, it may contain references to components of the currently rendered object. For \emph{declarative notations}, the presentation is computed from parameters such as fixity. The presentation has an optional output precedence.

The high-level structure of the presentation algorithm is as follows:
\begin{enumerate}
	\item Input: a presentable expression $E$ with optional {\mmturi} $U$, role $R$, and component list $C$, a style $N$, and an optional input precedence $\iPrec$.
	\item A notation $n$ applicable to $E$ is selected from $N$.
	\item The presentation $p$ of $n$ is obtained.
	\item If $R$ is bracketable, depending $R$, $\iPrec$, and the output precedence of $n$, $P$ is wrapped in brackets yielding $P'$.
	\item Output: the evaluation $\render{P'}{C}{0}$ of $P'$ in context $C$.
\end{enumerate}
All steps are described below.

\paragraph{Selecting Notations}
Styles may have multiple or no notations for the same notation key. The following rules are used:
\begin{itemize}
  \item A style may not locally declare two notations for the same key.
  \item A style has all locally declared notations, and all notations that imported style have, with one exception: Local notations shadow imported declarations for the same key.
  \item If $U$ is omitted, $n$ is the notation with role $R$ in $N$. It is an error if no such notation exists.
	\item If $U$ is given, then
	   \begin{itemize}
	    \item if $N$ has a notation for either $(U,R)$ or for $(-,R)$, then that $n$ is that notation,
	    \item if $N$ has notations for neither $(U,R)$ or $(-,R)$, it is an error.
	    \item if $N$ has notations $n_1$ with presentation $p_1$ for $(U,R)$ and $n_2$ with presentation $p_2$ for $(-,R)$ and $n_2$ has the \snippet{wrap} flag, then $n$ is a notation with presentation $p_2(p_1)$ (see below for applying presentations); the precedence of $n$ is that of $n_1$.
	    \item if $N$ has notations $n_1$ for $(U,R)$ and $n_2$ for $(-,R)$ but $n_2$ does not have the \snippet{wrap} flag, then $n=n_1$.
    \end{itemize}
\end{itemize}

\paragraph{Obtaining Presentations}
If $n$ gives a presentation directly, $P$ is that presentation.

Otherwise, $P$ is computed from the declarative parameters of $n$ as follows:
\begin{enumerate}
  \item The operator is the first component in $C$.
  \item According to the number $i$ of implicit arguments determined by $n$, the following $i$ components are the implicit arguments, the remaining components the explicit arguments.
	\item For pre- and postfix notations, the implicit and the explicit arguments are listed with the separator given by the fragment \snippet{argsep}. Then operator, implicit, and explicit arguments are used to fill the corresponding placeholders in the fragment \snippet{pre} or \snippet{post}, respectively.
	\item For infix notations, the operator and the implicit arguments are used to fill the placeholders in the fragment \snippet{operimp}. Then the result is placed among the explicit arguments using the fragments \snippet{opsep} and \snippet{argsep}.
\end{enumerate}


\paragraph{Bracketing}
If $R$ is not bracketable, no brackets are generated and $P'=P$.

Otherwise, the $P'=b(P,d)$ where $d$ is a below and $b$ is the presentation of one of the following three fragments: \snippet{fragment:brackets}, \snippet{fragment:ebrackets} (elidable brackets), or \snippet{fragment:nobrackets}. While these fragment names a fixed meaning, it is still the style's responsibility to provide notations for them.

The selection among the three fragments is as follows:
\begin{itemize}
	\item If $\iPrec$ is not given, no brackets. Thus, expressions can be forced to be unbrackets by not giving an input precedence. This is typical when recursing from structural levels into object levels.
  \item If $\iPrec$ is given, then the brackets depend on the difference $d=\oPrec-\iPrec$:
  \begin{itemize}
	  \item no brackets if $d=\infty$.
	  \item elidable brackets if $0<d<\infty$.
	  \item brackets if $d\leq 0$.
	\end{itemize}
\end{itemize}

\paragraph{Special Fragment Names}

The following table lists all fragment names that have a special meaning.

\begin{tabular}{|l|p{5cm}|l|}
\hline
Fragment  &  Placeholders & Function \\
\hline
\snippet{brackets} & bracketed expressions & unelidable brackets \\
\snippet{ebrackets} & bracketed expressions, elision level & elidable brackets \\
\snippet{nobrackets} & unbracketed expression & no brackets \\
\snippet{argsep} & none & separator between arguments \\
\snippet{opsep} & none & separator between operator and arguments \\
\snippet{argsep} & none & separator between arguments \\
\snippet{pre}, \snippet{post} & operator, implicit arguments, explicit arguments & pre- and postfix notations \\
\snippet{operimp} & operator, implicit arguments & operator in infix notations \\
\hline
\end{tabular}